\documentclass{article}

\usepackage[natbibapa]{apacite}
\usepackage{hyperref}
\usepackage{todonotes}

\title{Reliability and Validity}
\author{
	Barg, Julian\\
	\texttt{jbarg.phd@ivey.ca}
	\and
	Zbaracki, Mark\\
	\texttt{mzbaracki@ivey.ca}
}

\begin{document}
	\maketitle

	%!TEX root = ../main.tex

What is sustainable? What is sustainability? There are some formulaic definitions that are widely accepted, including the dogmatic definition that "[s]ustainable development is development that meets the needs of the present without compromising the ability of future generations to meet their own needs" \citep[IV]{WCED1987}. But if you ask two sustainability scholars to define sustainability in their own terms, chances are you are still going to get very different answers. Potentially including the ubiquitous phrase "I know it when I see it" \citep{White2013}. All that is not to criticize sustainability scholarship. It is the nature of the beast of dealing with complex systems--such as our planet--that things are difficult to nail down.

Definitions attain importance in research because they shape the research that takes place after the introduction section. Things may be assumed away, and a definition can determine the direction of a research project. The definition of sustainable development by the World Commission on Environment and Development (WCED)--despite being the most widely used definition of sustainability by far--is still contested. On the one hand, the WCED definition is widely used as a working definition. On the other hand, the WCED definition is criticized for putting too much focus on economic \textit{development} for resolving current and future distributive and environmental problems \citep[e.g.,][6f.]{Constanza2014a}.

This blurb introduces the discourse on the WCED definition and what assumptions it introduces to research. It covers on the one hand research in business journals that apply the definition. These articles have in common that they turn to businesses and their policy decisions for potential solutions to common environmental or social problems. On the other hand, there is a critical 
% \todo{Avoid label "critical" for its connotations?} 
discourse on sustainable development that explicitly discusses the inherent assumptions and typically rejects the notions that companies are suitable agents that can be relied on for change \citep[e.g.,][]{Banerjee2003}.
% This critical  discourse maintains that action by businesses is and will remain insufficient, and our current inaction outside the business realm will be costly in environmental and economic terms.

The discourse is driven by competing assumptions on the ability for corporations to act as agentic actors for change. 
\begin{itemize}
	\item \hl{Add "third way" implications similar to \citet{Giddens1979}}.	
	\item Actually in some sense, this directly parallels Gidden's comments on structure vs. agency--e.g., \citet{Heikkurinen2019}.
	\item the "third way" is actually the extant literature I have extracted before--\url{http://wiki.jbarg.net/Constructivism%20%26%20Ecology.md#environmental-science-stream}
\end{itemize}


	%!TEX root = ../main.tex

\subsection*{Scale}

Reliability and validity are applicable concepts for collective learning at any scale--be it at the level of a small organization or at the scale of our global society. On the same issue, one might observe a difference in validity and reliability of the learning process depending on the scale. For instance, the same discovery will have a different impact on collective knowledge at the level of the research lab where the discovery has taken place, at the level of the university, and at the national level.

\begin{itemize}
	\item The larger the organization (or collective) the more obvious the divergences. Most obvious at societal and global level.
	\item Managing divergences is a core feature of large organizations, such as corporations--think corporate culture. Employees might be quite cynical about the strategy of a corporation, but impression management may prevent that information to get out.
		\subitem Here, could mention that examples from sustainability are used--or could just move on and let the reader figure that out herself. 
		\subitem If wanted to make it explicit, reasoning would be that there is less need to disentangle truth from politics of truth--although there is still some.
\end{itemize}

	%!TEX root = ../main.tex

\subsection*{Validity}

% Correct predictions for the time being. Confusingly, correct predictions can occur even if the model is technically wrong--such processes are usually only discovered if later wrong predictions occur and a new valid model is created to rectify problems with the old model, that may have served well for a long time \citep{Madsen2010}.

Validity is the extent to which a learning process generates knowledge through which a collective can understand, predict, and control current and future events \citep{March1991a,Rerup2021}. The most illustrative example are explicit quantitative models that predict or allow to calculate specific values. 
% For instance, if a supermarket predicts \$19,000-\$21,000 in sales for toilet paper, and the actual sales end up being \$19,500, certainly we can say that the team went through a valid process to construct their confidence interval.
% \todo{Use "valid knowledge" here? Or continue to use "valid" and "reliable" only to describe processes?} 
% On the other hand, if then a global pandemic hits and a severe shortage in toilet paper occurs on a national level, we could then say that an invalid model of demand was constructed. That process would not be located at the level of the supermarket anymore, but the supermarket team's learning would have been invalidated.
For instance, \citeauthor{Manabe1967} predicted in \citeyear{Manabe1967} that a doubling of carbon dioxide in the atmosphere would lead to a temperature increase of 2.3°C--a value that was later validated by observations \citep{Forster2017}. Their efforts--in conjunction with the work of predecessors and research lab at Princeton--undoubtedly represents an example of valid organizational learning.

\begin{itemize}
	\item Unreliable learning
		\subitem \citet{Guarino2020}
		\subitem Predictions of sea-ice loss to conservative
\end{itemize}

	%!TEX root = ../main.tex

\subsection*{Reliability}

\begin{itemize}
	\item Reliable learning
	\item Unreliable learning
\end{itemize}

	%!TEX root = ../main.tex

\subsection*{Quadrants}

Similarly populate this section with examples, without drawing attention away from purpose of introducing terminology.

\begin{itemize}
	\item Expert/Technicist learning
		\subitem Low reliability \& high validity
	\item Popular/Populist learning
		\subitem High reliability \& low validity
	\item Regulatory/Political learning
		\subitem High reliability \& high validity
		\subitem Building coalitions around (selected) pieces of valid learning
	\item Skeptical learning
		\subitem Low validity \& low reliability
\end{itemize}

	%!TEX root = ../main.tex

\subsection*{Discussion}

\begin{itemize}
	\item Some observations of (political) processes that complicate processes of reliable and valid learning. Below some examples.
	\item \citet{Pailler2018}:
		\subitem Acting against better knowledge when (individual) political interests involved
	\item \citet{Aronczyk2019}:
		\subitem Interest groups insert themselves into learning processes early on, with complex implications for knowledge creation
		\subitem In this case, the interest groups has engaged in valid learning on stakeholders (MNCs) and their interests
			\subsubitem Or reliable learning, because their prediction on industry action did not pan out? Very political process at hand.
		\subitem Lessens primacy of validity with regard to \textit{physical} processes--the \textit{social} sphere gets to be considered
	\item \citet{Boudet2020}:
		\subitem "Resilience" of reliable over valid learning. Reliable learning can persevere in the face of overwhelming evidence that \textit{should} constitute cosmology episode
\end{itemize}

	% \section*{Reliability and validity in organizational learning}

	% This section provides a brief literature review of the existing work on reliability and validity.

	% \section*{Methods}

	% This section introduces which literature was used to develop the ideas that are introduced subsequently.

	% \section*{Findings}

	% This section works twofold. First, it lays out the processes of (un)reliable and (in)valid learning in clear terms. Then, the material from the methods section are used to show the processes in action and develop additional concepts. For instance, provide examples of invalid but reliable learning on a societal or organizational scale and show how the situation came to be. This section will be developed through additional tables on the literature that was introduced in the methods section. I have previously developed some of these tables, but have sighted additional literature with a critical view that now needs to be integrated (see \url{http://wiki.jbarg.net/Constructivism%20%26%20Ecology}).

	\section*{Conclusion}

	Just a summary? Spelling out implications for sustainability seems construed? Implications for different fields maybe?

	\clearpage
	\bibliography{bibliography}
	\bibliographystyle{apacite}

\end{document}