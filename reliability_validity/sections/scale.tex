%!TEX root = ../main.tex

\subsection*{Scale}

Reliability and validity are applicable concepts for collective learning at any scale--be it at the level of a small organization or at the scale of our global society. On the same issue, one might observe a difference in validity and reliability of the learning process depending on the scale. For instance, the same discovery will have a different impact on collective knowledge at the level of the research lab where the discovery has taken place, at the level of the university, and at the national level.

\begin{itemize}
	\item The larger the organization (or collective) the more obvious the divergences. Most obvious at societal and global level.
	\item Managing divergences is a core feature of large organizations, such as corporations--think corporate culture. Employees might be quite cynical about the strategy of a corporation, but impression management may prevent that information to get out.
		\subitem Here, could mention that examples from sustainability are used--or could just move on and let the reader figure that out herself. 
		\subitem If wanted to make it explicit, reasoning would be that there is less need to disentangle truth from politics of truth--although there is still some.
\end{itemize}