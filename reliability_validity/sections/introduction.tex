%!TEX root = ../main.tex

The real world does not listen to scientists. At least not primarily. There could be examples about climate change here, but more than a year of living with Covid have made it sufficiently clear that decision makers first make the decisions and then ex-post find a way of explaining how those decisions fit with the science. Fortunately for them, scientists in official positions--think Fauci--often look for ways not to embarrass their bosses too much even if they correct them.

As a result, the world has gotten difficult to navigate. On the one hand, the 21st century so far has been the final nail in the coffin for the modernistic notion of progress (there is a reason why people don't often use the term third millennium anymore). On the other hand, even Bruno Latour eventually felt the need to reconcile between the obligation to remain critical in every regard, and the recognition that some discoveries--such as climate change--are "matters of concern" \citep{Latour2004}. So how do we reconcile the fact that--to stay with the example of Covid--scientists were able to sequence the DNA of the virus in a matter of weeks and develop a vaccine within just a few months, but collectively the world still suffered millions of deaths?

Reliability and validity of knowledge are the factors that explain a divergence between organizational and collective learning (such as scientific research) and real world outcomes. This article lays out a theory of reliability and validity of organizational learning. Validity learning describes processes that allow for prediction and control of outcomes. Reliability is given when the learning outcome is also public, stable and shared \citep{March1991a}. 

% From here on, we depart the convenient example of Covid and focus on the more extensive discourse on organizations and the natural environment. Specifically, this article uses the discourses on environmental sustainability and real-world environmental impacts to show the dynamics of reliability and validity on a large scale. The discussion section brings the debate back to the organizational level by presenting a model and some hypothetical examples.