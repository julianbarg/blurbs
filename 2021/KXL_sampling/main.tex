\documentclass{article}

\usepackage{xcolor}
\usepackage{soul}
\usepackage[natbib=true, style=apa]{biblatex}
\usepackage{tabularx}
\usepackage{booktabs}
\usepackage{parnotes}
\usepackage{hyperref}

\title{KXL Sampling Strategy}
\author{
	Barg, Julian\\
	\texttt{jbarg.phd@ivey.ca}
}

\sloppy

\addbibresource{bibliography.bib}

\newcommand{\f}{&&&&}
\newcolumntype{s}{>{\centering\arraybackslash}m{1 cm}}
\renewcommand\tabularxcolumn[1]{m{#1}}% for vertical centering text in X column

\begin{document}
	\maketitle

	\newpage

	The goal of this study is to understand the decision making process on Keystone XL. For that purpose, we retrace the flow of information, in particular including bottlenecks. Different epistemic communities contribute information in different areas, such as engineering, ecology, economics, etc. By drawing up the general epistemologies at play, we obtain a model that contains most relevant information without the unwieldy, minute details.

	Over the course of 12 years and counting, TransCanada is in an ongoing battle to obtain a permit for its Keystone XL pipeline. The decision to build the Keystone would have major environmental consequences, both for the local and global environment. We are after a general process model how collectives make such major decisions that shape our--physical or social--environment. The overall decision making process is broken up by individual events where formal decisions are made. Although the decisions are not \textit{actually} made in those very moments--i.e, the decision to be made may be obvious for some or many participants ahead of time--for the purpose of sampling it is useful to attend to these events. I refer to these events as \textit{decision making events} to differentiate them from the \textit{decision making process} that we theorize about.

	We begin by mapping out the epistemic communities at play. To obtain the list in a reproducible fashion, we take a four-step approach. \textit{First}, we map out a general history of the decision making process on Keystone XL. On Factiva, we obtain all news articles on Keystone XL\footnote{Using the keywords "Keystone XL" and "KXL".} in \hl{[list of American news outlets]} from when the pipeline was first announced on \hl{[date]} to \hl{[end of sampling period]}. We sight the articles and note all decision making events where the different epistemic communities come together to engage in decision making. In the \textit{second step}, we obtain archival material on the decision making events and note down all organizations that partake for each of them. See \autoref{tab:SD} for an example.

	\begin{table}
	\caption{Parties in decision making processes}
	\label{tab:SD}
	\begin{tabularx}{\textwidth}{s s s s X}
		Year (start)	
		& Place 		
		& Docket etc. 
		& Qualifier 
		& Institution/Qualifying party\\ 
		\toprule\\

		2014 
		& South Dakota 
		& 14-001 
		& Party status 
		& South Dakota Public Utilities Commision\parnote{\url{https://puc.sd.gov/commission/orders/hydrocarbonpipeline/2014/hp14-001interv.pdf}}\\\\

		\f TransCanada\\\\
		\f Rosebud Sioux Tribe -- Tribal Utility Commission\\\\
		\f Bold Nebraska\\\\
		\f Standing Rock Sioux Indian Tribe\\\\
		\f Rosebud Sioux Tribe\\\\
		\f South Dakota Wildlife Federation\\\\
		\f Cheyenne River Sioux Tribe\\\\
		\f Sierra Club\\\\
		\f 350.org\\\\
		\f Yankton Sioux Tribe\\\\
		\f Dakota Rural Action\parnote{Including whistleblower Evan Vokes.}\\\\
		\f Indigenous Environmental Network\\\\
		\f Intertribal Council on Utility Policy\\\\
		\f Local residents\parnote{In additional public hearings.}
	\end{tabularx}
	\parnotes
	\end{table}

	After obtaining a list of the participants in the decision making process, we further close in on the epistemic communities at play in the \textit{third step}. We prepare an exposé on each of the organizations that participate, including some background information and their role in the decision making event or events. \hl{An arbitrary number of } coders then read through the exposés and categorize the participants according to the epistemologies that the participants bring to bear. In an iterative process, we ask the coders to address differences in coding until we arrive at a consistent classification system. These are our epistemic communities for the next step.

	In our \textit{last step}, we select prominent members of the epistemic communities that we have addressed in the previous step and collect the documents to fall into our data sample. The material we sight consists of four parts. (A) Communication of the epistemic community on the decision making process and individual events. For instance, Bold Nebraska operates a blog that comments extensively on the decision making process and individual events. (B) Public documents on the decision making events such as testimonies and rebuttals which reveal the discourse between the different epistemic communities. (C) Where any of the material hints at a relevant conversation taking place outside the previous two channels, we obtain those documents also.\footnote{For example, the TransCanada whistleblower Evan Vokes testified before the Senate of Canada, which we utilize as an example of an engineering community conversing with the regulatory community.} (D) To achieve convergence and triangulate evens, we carry out interviews with representatives of each community where we obtain more data on the discourse and the participants epistemologies.

	% Mapping out all epistemic communities and the knowledge they bring to bear on the decision making process, allows for the creation of a simple model of the decision making process without the need 

	% \clearpage
	% \printbibliography
\end{document}