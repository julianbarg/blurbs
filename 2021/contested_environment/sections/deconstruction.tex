%!TEX root = ../main.tex

\medskip

% \section*{Structure}

% \subsection*{Deconstructing an environmental threat}
\section*{Deconstructing an environmental threat}

% \subsubsection*{Approach} 
\subsection*{Approach} 

\begin{itemize}
	\item Only demonstrate deconstruction
		\subitem Equivalent to construction of pipelines as safe
		\subitem Equivalent to statement of healthy, unaffected nature
	\item Focus on industrial side of action
	\item Deliberately one-sided view
	\item Using greenwashing terminology
	\item Contribution: Greenwashing as a discourse strategy
		\subitem Specific acts of greenwashing
		\subitem Cataloging different kinds of greenwashing
		\subitem Greenwashing at different levels
	\item Contrasting claims and industry/organizational trajectories
\end{itemize}

% \subsubsection*{Data points--events}
\subsection*{Data points--events}

\begin{itemize}
	\item Keystone XL permit application \citep{Stansbury2011}
	\item AOPL project to advertise pipelines as safe
	\item EPA environmental restoration projects
	\item ALEC anti-protest laws \& attack on legitimacy of anti-pipeline protesters
	\item Spending rhetorics--pipeline safety spendings on pipeline extensions
	\item Pipeline technology
	\item Escalation through use of police, private security, \& national guard
	\item Astroturfing
\end{itemize}

% \subsubsection*{Contribution}
\subsection*{Contribution}

Introduce the various industry projects as more than just initiatives to disseminate information. The "information dissemination"-premise suggests that recipients receive information from various sources and when they are due to make a decision, they search out more information to be considered in the decision-making process. That view would be in line with rational models of decision making. For instance, a politician might consult environmental data, too, and conduct a critical examination of the information that he\footnotemark[\ref{note1}] has received before.

What makes a greenwashing strategy so insidious and effective is the model of decision making that it suggests. Many of the information provided in a greenwashing campaign are often "harmless", as in easy to debunk through independent research. But when an alternative model of decision making is considered, a disinformation campaign suddenly appears quite effective. In this model, the decision maker has already formed his opinion on a matter when he\footnotemark[\ref{note1}] is called upon to make a decision. Any new arriving information then is compared with the existing information, and when the decision maker encounters information that contradict previous information, he can either decide to invalidate those information--e.g., based on the sender--\textit{or} reconsider the already formed opinion.
