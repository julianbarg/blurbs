%!TEX root = ../main.tex

\section*{Epistemic communities}

Organizational theory and management research often pride themselves with being at the intersection of research and practice, while at the same time bemoaning--or taking refuge behind--the researcher-practitioner gap when asked about making positive contributions to practice \citep{Kieser2015}. Researchers \textit{want} to help practitioners to make better decisions, but they \textit{don't want} to be tethered to "people who seem mentally challenged when reading the Harvard Business Review" \citep[p. 823]{McKelvey2006}. Their solution is to find mechanisms in the data that take over the task for researchers, to find rationality--or more recently bounded rationality--in the world.

Examples of the search for rationality or bounded rationality in the wild include the \textit{theory of the firm}, \textit{transaction costs}, and the broader literature on \textit{dynamic capabilities}. That is not an exhaustive list--almost every contemporary article in the literature is showcasing a mechanism toward performance improvements, or some other kind of desirable organizational outcome. In other words, management research is telling practitioners "keep doing what you are doing, you are already doing pretty well", while actively looking for empirical evidence for that hypothesis. That is concerning, when there are other models of the world which highlight the counterforces that exist  \citep{Eisenhardt1992}. These are models that gnaw at the primacy of rationality and bounded rationality in organizational processes and that our findings should be weighted against.
% That is concerning, when what we would actually want scholars to do--if we accept that these are the concepts to work one--is to take a critical look at their hypotheses, the counterfactuals, and effect sizes, in order to critically examine whether our models of the world are suitable, functional "speculations of the world".
Is selection pressure really a sufficient force to bring about all the hypothesized processes? How does selection pressure fare against organizational discontinuities in practice and personnel and other organizational or meta-organizational processes?

\subsection*{Further}

\begin{itemize}
	\item Pivot to epistemic communities 
		\subitem As a model to see shortcomings
		\subitem As a model to see the construction of a (seemingly) rational reality
		\subitem As a way to experience the world like practitioners do
		\subitem As a way to get an elevated perspective
		\subitem In the end, can still analyze the merit of different views
\end{itemize}

\subsection*{Problem}

\begin{itemize}
	\item How to integrate this with purpose of showcasing contestation, competing epistemologies
\end{itemize}