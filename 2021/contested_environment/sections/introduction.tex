%!TEX root = ../main.tex

Is the natural environment real or socially constructed? That is a presumptuous question to ask, but it is a great one if you want to draw in your reader. In short, our perception of the environment needs to be sufficiently accurate for our species to survive a process of natural selection--meaning that our immediate physical environment do probably exists independent of us in a fashion akin to how we perceive it \citep{Hoffman2015a}. The further we venture from our immediate environment in our inquiries however, the more we need to rely on the perception of others, who co-construct with us a larger picture of our environment. Even if we were to travel far and wide, we simply could not observe by ourselves all the details of all the various ecosystems, from the trees and animals herds down to microbes and microplastics. Even less could we hope to monitor them over time without the help of others. Some of the greatest collectives on earth are dedicated to this task of observing earth, such as NASA and the science community NASA cooperates with.

In our quest to understand the world, we encounter those that agree with us, and--if we look well enough--those who do not agree with us. In order to create an encompassing, widely accepted understanding we need to create a model that can accommodate diverging views with more elegance than just stating "this side is wrong and that side is right". In particular, it may be tempting to disregard the opinion of those who do not have a good stewardship of nature in mind. But ignoring a person does not change the fact that the person exists\footnote{Some might argue that that is unless you get everybody in society to agree to this--see homelessness.}, and that their actions have an impact on our environment--and often a more significant one than we do. In this work, I conduct a deep dive into one environmental phenomenon, to obtain a more encompassing view into the social processes involved.

One of the actors in this quest is nature itself. The title already gives away that I look at the construction and deconstruction of an environmental threat. That implies that it is in man's\footnote{\label{note1}Something, something, most people in the industry being male etc.} power to make an environmental threat appear and disappear. However, that is not the full story. As long as the threat is speculative, it does hold true. But every now and then, the speculative threat manifests in an actual environmental pollution event. When that happens, an army of men\footnotemark[\ref{note1}] descends upon the site of the environmental damage and spins more than an Olympic figure skater. But for a limited period of time--while the police sets up their barriers and tells you that there is nothing to see there--it is there, for everybody to see. And the general public will see it--whether it is oil in their backyard, a river on fire, or maybe an explosion that kills three.\footnote{In a similar vain, but on a different analytical level, \citeauthor{Berger1966} spoke of the "dialectic between nature and society" (\citeyear[p. 201]{Berger1966}).}

\begin{itemize}
	\item I am very happy with this introductory paragraph. It introduces three things very effectively:
		\subitem The social nature of (our mental image of the) extended environment
		\subitem The sometimes deliberate nature of this social process
		\subitem The dialectic nature of this relationship
	\item One thing is still missing, which is how well March fits into this:
		\subitem March's political view of organizations would fit quite nicely, except that it is here the scope is bigger--the same process happens on a smaller scale, too, but on a bigger scale it becomes more salient
		\subitem Validity is of course also relevant, which may be where the actual contribution lies
\end{itemize}