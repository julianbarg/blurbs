\documentclass{article}

\usepackage[natbib=true, style=apa, alldates=year]{biblatex}
\usepackage[pdfauthor={Julian Barg},pdftitle={OMT prep}]{hyperref}


\addbibresource{bibliography.bib}

\title{Discourse and Institutions: Why Keystone XL Was Approved, and Why It Was Not}

\author{
	\texttt{jbarg.phd@ivey.ca}\\
	Ivey Business School
}

\sloppy

\begin{document}

	\maketitle
	\clearpage

	\begin{quote}
		Current public debate about oil-sands development focuses on individual pipeline decisions. Each is presented as an ultimatum — a binary choice between project approval and lost economic opportunity. This approach artificially restricts discussions to only a fraction of the consequences of oil development, such as short-term economic gains and job creation, and local impacts on human health and the environment. Lost is a broader conversation about national and international energy and economic strategies, and their trade-offs with environmental justice and conservation.

		-- \citet[p. 466]{Palen2014}
	\end{quote}

	The realization must have been painful. For half a century, climate scientists has worked on more and more intricate models to understand and predict climate change. The effects of a ton of carbon have been well understood for decades \citep{Forster2017}. Only to discover that economic interests in fossil fuels are so entrenched that they could barely get a word in in the debate on a mega project like Keystone XL. For social scientists, and organizational in particular, the difficulty to rein in on entrenched economic interests should not come as a surprise--scientists have encountered a phenomenon that we know as an "institution".

	

	Debate \textit{can} lead to deinstitutionalization. 

	% An account of a functional discourse is provided by \citet{Maguire2009}. In 1962 Rachel Carson published \textit{Silent Spring}. The bestseller about the dangers of the pesticide DDT was initially met with skepticism, but it has also been credited with bringing about the deinstitutionalization of DDT. \citeauthor{Maguire2009} describe both the initial push-back against Carson's work, and how the book revived the research on DDT. The new research was translated into regulations, and DDT eventually outlawed.

	% This research takes \citet{Maguire2009} as a starting point, but the difference of the nature of the discourse requires some adjustment to the methods applied.

	% They start from event. I do not.

	% This is not the place to talk about the climate impacts of the project. They have been written about and--unfortunately--they do not show up much in the discourse outside of routinized references.

	\clearpage
	\section*{References}
	\printbibliography[heading=none]

\end{document}