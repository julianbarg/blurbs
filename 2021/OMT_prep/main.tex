\documentclass{article}

\usepackage[pdfauthor={Julian Barg},pdftitle={OMT prep}]{hyperref}
\usepackage[natbib=true, style=apa, alldates=year]{biblatex}

\addbibresource{bibliography.bib}

\title{Discourse and Institutions: Why Keystone XL Was Approved, and Why It Was Not}

\author{
	\texttt{jbarg.phd@ivey.ca}\\
	Ivey Business School
}

\sloppy

\begin{document}

	\maketitle
	\clearpage

	% \begin{quote}
	% 	Current public debate about oil-sands development focuses on individual pipeline decisions. Each is presented as an ultimatum — a binary choice between project approval and lost economic opportunity. This approach artificially restricts discussions to only a fraction of the consequences of oil development, such as short-term economic gains and job creation, and local impacts on human health and the environment. Lost is a broader conversation about national and international energy and economic strategies, and their trade-offs with environmental justice and conservation.

	% 	-- \citet[p. 466]{Palen2014}
	% \end{quote}

	\begin{quote}
		Keystone XL shutdown may signal the end of major U.S. oil infrastructure

		-- \citeauthor{Freitas2021} for \textit{World Oil} (\citeyear{Freitas2021})
	\end{quote}

	\begin{quote}
		The fossil-fuel industry’s aura of invincibility is gone. They’ve got all the money on the planet, but they no longer have unencumbered political power. Science counts, too, and so do the passion, spirit and creativity of an awakened movement from the outside, from the ground-up.

		-- \citeauthor{McKibben2015} for \textit{The Guardian} (\citeyear{McKibben2015})
	\end{quote}

	To say that Keystone XL is a controversial issue would be an understatement. The commentators, be it analysts, politicians, scholars, or activists, typically fall into one of two camp. The first group focuses on the economic benefits of the project. With regard to environmental concerns, they emphasize recent technological advancements, TransCanada's willingness to go the extra mile with regard to safety, and an optimistic, forward-looking view of the modern American economy. The other group of commentators is weary of the structural risks incurred by a project of such scale. They question the economic benefits, raise the risks associated with a worst-case spill, and draw attention to vulnerable local ecosystems--in particular water resources.
	
	% Commentators on environmental issues, scholarly or not, generally fall into one of two categories. There are those who view businesses as actors who can tackle environmental issues, or who benefit from improving environmental performance. And there are those who take a critical approach. These commentators emphasize structural issues and hurdles to improvements. Scholars of organizational theory will recognize this dichotomy as a question of structure versus agency.

	% Over the last 30 years, the business community and business scholars have taken up a great interest in environmental impacts, and environmental sustainability. There are two schools of thought on the issue. At one end of the spectrum, there is research that focuses on businesses as actors who tackle environmental issues, or who benefit from improved environmental performance. On the other end of the spectrum, there is a critical approach, which emphasizes structural issues, and hurdles to improvements.

	% At the heart is an enormous amount of environmental data that has been produced by businesses, for businesses. On one side of the chasm, this abstract data is assembled into neat insights on business and environmental impacts, or investor attitudes on the environment \citep[e.g.,]{Flammer2017,Yan2021}. On the other side of the chasm, there is a stream of literature that disassembles and question the validity of the metrics and indicators \citep[e.g.,]{Chatterji2009,Chatterji2016,Eccles2020}.

	% Organizational theorists \textit{have} previously circumnavigated the chasm. Before the \textit{business case for sustainability}, and before ESG indicators had become a mainstay in the business literature, institutional theory and qualitative methods were routinely used to explore environmental phenomena \citep[e.g.,]{Hoffman1999}. Where social action is based on values and beliefs, on what is \textit{legitimate}--as is the case for environmental impacts--institutional theory is uniquely suited for the task. \citet{Maguire2009} went one step further by selecting a case study with clear, \textit{positive} environmental implications. Their research on DDT showcases how environmental phenomena can benefit the discourse on institutional theory, when values and beliefs as well as environmental outcomes are salient in the empirical context. Another avenue is longitudinal, participatory research. Unfortunately, due to the trajectory of global emissions, this area of research is destined to find a rather limited set of negative outcomes \citep[e.g.,]{Wright2017}. As a result, research that applies institutional theory typically falls into two camps: studies that explain \textit{unsustainability} \citep{Ergene2020} or studies on cases that are selected for their noteworthy positive outcomes.

	% An alternative to those approaches is to focus on smaller episodes, and to select multiple over time.


	% Not much was at stake, because alternative already existed. The industry could acquiesce.

	% \citet{Maguire2009} tackle the deinstitutionalization of DDT--a pesticide with clear environmental impacts. 

	% This research represents an effort to circumnavigate the chasm by focusing research on a \textit{Ding} \citep{Latour2005}--a specific instance or source of environmental impacts. Institutional scholarship \textit{has} previously 

	% the opportunity afforded by Keystone XL is that

	% remains abstract

	% If carefully selected, the analysis of a \textit{Ding} generalizes nicely to the \textit{modus operandi} of how environmental impacts come about. The Keystone XL debate affords us a look at the reality of environmental impacts without necessitating a discussion of what is real, and what is misleading--or \textit{hyperreal}. 



	\clearpage
	\section*{References}
	\printbibliography[heading=none]

\end{document}