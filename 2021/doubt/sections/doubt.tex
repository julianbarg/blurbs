%!TEX root = ../main.tex

\section*{You wanna hear my theory on it?}

Rather than giving organizations the benefit of doubt, I want to talk about the benefits of sowing doubt. The theory in this domain is torn. Where companies engage in greenwashing, the literature looks for the organizational complexities that allow companies to keep a straight, clean face while cooking meth in their backyard \citep{Bromley2012}. As scholars, we could not possibly make an indictment of misconduct where the facts are not 100\% clear. \citet{Whiteman2016} are the exception that proves the rule--the article touches on the number of hoops the authors had to jump through. On the other hand, misconduct is another literature entirely \citep{Greve2010}. When a criminal indictment exists, suddenly we are allowed to speculate about the nefarious motives. If an organization is criminal, it is fair to assume that nefarious motives exist. And yet even then, OT sometimes gives organizations the benefit of doubt and speaks of legal gray areas \citep{Mohliver2019}.

What about those cases where a criminal indictment does not exist, but it can be seen from a mile away? Are scholars really more oblivious than investigative journalists, activists, and large swaths of the public with regard to morally reprehensible behavior by organizations? I set the tone in the \nameref{sec:intro}, but there is no doubt that any of us can think of an  example of an organization where everybody saw it coming. To make it more clear: there are obvious criminal organizations with nefarious motives in this world--think mafia, bank robbers, and gangs chasing "clout". These organizations exist around the globe, and they have diverse forms, each connected to local culture. Why should nefarious motives not exist in commercial operations? Crime is not a blue collar phenomenon.

% What we observe is an organization with only one possible favorable outcome that is pushing hard toward that outcome, and utilizing legal gray areas. 

\subsection*{How to effectively sow doubt and reap the rewards}

Sowing doubt can be very lucrative--and it is the only pathway for certain industries. Some, such as the Koch Brothers, even seem to have made a business model out of it. One can assume that organizations accidentally venture into gray areas (see \nameref{sec:lit}), but that would underestimate organizations agentic decision making ability. If a gray area exists, \textit{why shouldn't an organization search out this space?} The tobacco industry, the oil industry, and the arms industry do not lobby to establish facts--they revel in the gray area on health impacts, climate change, and their role in international conflicts.

What's the playbook? Based on the case of Taylor Energy, and the initial environmental impact assessment that was commissioned by TransCanada, and the networks of climate change deniers, I suggest the following. Discourse is an opportunity to venture into the gray area. To embark, the original author does not have to make an airtight argument. Obfuscation and doubt already has a chance to succeed when the author makes an argument that is robust to some superficial scrutiny. The chance of success depends on the efforts expanded by the author, and those expanded by the audience. Note that a good-willed audience might be inclined to expand less efforts, so network characteristics matter to this game of charades, too.

In the process of generating doubt, co-members of an organization or network allies such as consulting firms are an internal audience. The same rules of efforts expanded and goodwill apply to these co-members: for a message of obfuscation to be officially adopted by an organization, it does not need to be shared and understood by all involved co-members. Rather, it just needs to be robust to different levels of scrutiny. Take for instance the network inside the Cato Institute, which was founded as the Charles Koch Foundation and frequently engages in climate change denial. Jerry Taylor says of his time inside this organization that individuals making bad faith arguments were known by their colleagues for doing so, and that he and others refused to pick up those dubious arguments. Instead, Jerry Taylor relied on economic arguments that built on uncertainty and confidence intervals, and were much more difficult to disentangle. However, more nefarious individuals were still able to make their dubious arguments under the banner of the Cato Institute, suggesting the existence some institutional support despite internally voiced complaints from other members of the organization. In other words, just some co-members or even none of them need to be co-conspirators.

Another mechanism that allows nefarious messages to diffuse through a network is plausible deniability. Where an actor recognizes a specific rhetoric \citep[cf.][]{Zbaracki1998} as useful in a discourse, but is on the fence as to whether the rhetoric will hold up to scrutiny, the actor might choose to adopt the rhetoric if it is reasonably certain that the actor can avoid liability or accountability.

Casting doubt has a lot of benefits. Sometimes, the mere existence of a counterposition gives an organization a fighting chance. As mentioned above, depending on the network constellation other actors might be willing to adopt a rhetoric of a nefarious actor without much scrutiny, or without any scrutiny at all. The purpose of casting doubt also is not necessarily to "win" in a discourse once and for all. Rather, the time it takes for other actors (e.g., the court) to scrutinize an argument may buy enough time for the focal actor to hold out until the network constellation changes in the actors favor. For instance, there might be a regime change, and a political leader might be willing to change the executive and judicial system in the actors favor. In other cases, an actor might be caught up in a desperate fight for survival, and willing to grab at straws.