%!TEX root = ../main.tex

\section*{Introduction}\label{sec:intro}

In 2004, Hurricane Ivan hit the Louisiana coast, and caused massive damages to the offshore oil industry. Most notably, an oil platform at the Mississippi River Delta was knocked over. Since then, the Taylor Energy oil platform has been continuously spilling oil into the Gulf of Mexico. At an estimated spill volume of 1.5-3.5 million barrels as of 2018, the spill rivals the 4 million barrels that Deepwater Horizon spilled into the Gulf in 2010.\footnote{\url{https://www.washingtonpost.com/energy-environment/2018/11/20/coast-guard-orders-cleanup-massive-year-oil-spill-gulf-mexico/}} Originally, after plugging some of the wells below the platform, Taylor Energy estimated the platform to only spill about two gallons of oil into the Gulf per day. However, in the aftermath of the Deepwater Horizon oil spill, Healthy Gulf--an NGO--discovered that the oil spill was much larger in scale. Aerial photos and videos show an oil slick a few miles long. Eventually, an independent contractor installed a containment system that collects about 1,000 gallons of oil per day.\footnote{\url{https://youtu.be/ztT45A501Tc}\label{foot:vice}}

For Taylor Energy, the original communication strategy held a lot of promise. After Hurricane Ivan toppled over the platform, Taylor Energy ceased to exist as an energy company. Only one permanent employee remained. However, Taylor Energy had been obliged to create a \$400 million fund for the clean-up cost--money that could potentially be returned to investors. As of 2021, Taylor Energy is suing the coast guard, the federal government, and the Couvillion Group--which contained the spill--for return of the money. Taylor Energy's efforts to close the case were not haphazard. In 2013, the organization convinced the participants of a joint workshop between three government agencies that capping the spill would likely result in more environmental damages than leaving the spill be \citep{Staves2013}. And in 2018, a PhD graduate of MIT issued a report stating that the oil sheen could conceivably have resulted from previously spilled oil being released by sediment, rather than actively spilling from the wells \citep{Camilli2018}.\textsuperscript{\ref{foot:vice}}

Taylor Energy's strategy is brazen, to put it mildly. There is a large disconnect between the large oil slick and the estimated spill volume of two gallons per day. Taylor Energy is suing the very company that captured video footage of the oil spill at the bottom of the ocean. In the words of Timmy Couvillion, CEO of the Couvillion Group "It's a joke".\textsuperscript{\ref{foot:vice}}

% Greenwashing describes the strategic use of misinformation \citep{Delmas2011}, or half-truths \citep{Lyon2011} for reputational gains in the environmental arena \citep{Bansal2004,Lyon2011}. 