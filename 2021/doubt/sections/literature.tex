%!TEX root = ../main.tex

\subsection*{More than decoupled from reality}\label{sec:lit}

Certainly, the Taylor Energy case meets the definition of "greenwashing" \citep{Lyon2015}. The case represents an agentic effort to make an oil spill disappear--an effort that was successful for over a decade. Yet, the common analytical OT lenses of greenwashing are not a great fit with this phenomenon \citep{Delmas2011}. There is no ambiguity here as to what the goals and ends are \citep{Crilly2012,Bromley2012,Wijen2014}. The process could be described as ceremonial conformity, but that would be an understatement. \citet{Meyer1977} propose that firms adopt certain practices because these practices are considered rational, and that decoupling is indicative of how the difficulty substantive action. Or in another version, an organization may adopt a practice because it sees the promise, but the practice loses its well-defined attributes when it is adopted to an individual organization with specific needs \citep{Zbaracki1998}. Other common theoretical lenses include selective disclosure \citep{Lyon2011}, and signaling theory \citep{Delmas2010}. Signaling would be the closest to what Taylor Energy is doing--but it does not capture how brazen the organization operates.
