\documentclass{article}

\usepackage[natbib=true, style=apa]{biblatex}
\usepackage{hyperref}

\addbibresource{../bibliography.bib}

\title{Reliability and Validity--Structure}

\author{
	Barg, Julian\\
	\texttt{jbarg.phd@ivey.ca}
}

\begin{document}

	\maketitle

	New suggested title: Learning at the level of the organization, population, and above

	Or: Organizational and collective learning

	\section*{Organizing \& progress}

	Reliability and Validity is doing something that is very similar to \citet{Levinthal1993}.\footnote{\citet{March1991a} follows a similar structure.} \citeauthor{Levinthal1993} are exploring the implications for organizational learning under the conditions of bounded rationality. More or less indirectly, this approach challenges the notion of organizational learning guaranteeing organizational progress \textit{in perpetuum} and shows other potential outcomes. At an early point, the authors execute what is somewhat of a pivot \citep[p. 97]{Levinthal1993}--although it serves the purpose of the article well. Here, the authors start to discuss exploration and exploitation. In short, the authors suggest that exploitation of existing capabilities allows organizations to gradually get better at what they are currently doing. But to focus on exploitation means potentially disregarding attractive alternatives that are unlike the current approach. Exploration of new options on the other hand may or may not fulfill that promise--but it also means the abandonment of expertise that has been built over time. Actors can only guess as to what direction it is ideal to direct their efforts toward. What's more, exploration and exploitation presents us with a mechanism of how we can get from a more desirable to a less desirable state, all while presumably learning. An organization could have a sophisticated production process in place before switching to a presumably better--but maybe just different--production process or to a slightly modified product. If the presumed advantages may never materialize, while an existing expertise is also lost. In summary, when an organization engages in learning, at the same time performance may improve or slump--we cannot know based on the observation of organizational learning alone.

	March's "criticism" of learning--if one can call it that--is quite poignant. It allows for the possibility that learning leads to progress. But it does not observe learning as indisputable evidence of progress. As such, is squarely in line with textbook postmodern thinkers who avoid theorizing in "grand narratives". These grand narratives usually constitute stories of social or technological progress that imply a development of humanity toward a--dystopian or utopian--end state based on individual concepts  \citep{Lyotard1984}. These concepts are then traced back across history to account for the major developments of humanity (often starting from the period of enlightenment). For instance, the concept of \textit{creative destruction} appears to imply that products and services get better and better, and hence so would our lives. That concept does not account for the different dimensions of such offerings. Rather than theorizing in grand narratives, social science needs to account for discontinuities, for a constantly changing social environment that does not develop in any particular direction over the long run \citep{Habermas1973}.

	Let's briefly apply this to organizations as well as the products and services they offer. \textit{Creative destruction}-type stories are a staple in the business literature. A more encompassing perspective would draw attention to idiosyncratic events, such as the early success and demise of the electric vehicle around the turn of the 20th century. Or the central role of the US Department of Defense's ARPANET--rather than any commercial corporation--in the development of the biggest disruptor of them all, the Internet. And finally, all those products--foodstuff, cars, and presumably appliances--that have become more affordable over the years but that are said to have "gone to shit". Or like Coca Cola, prices and sales can skyrocket even for a product uses a cheaper ingredient and is said to taste worse.

	\citeauthor{Levinthal1993} provides a perspective on learning that is limited to organizations and vicarious learning, but we can expand on what they have given us and arrive at the level of the population and above. Throughout his career, March was quite careful about suggesting the existence of an organizational intelligence that is more than an adaptive force. He embraced the concept of an adaptive intelligence. Organizations adopt to challenges as they emerge--\textit{or they disappear}! Organizations \textit{can} rely on their experience to scale hurdles that they have encountered before. Over time they become quite adept at that--even under conditions of turnover and resource scarcity they can maintain their ability to overcome familiar hurdles \citep{Cyert1963}. And when an they encounter a new challenge, organizations have the ability to look far and wide, and to overcome this new challenge by adapting lessons from their own history, or near-history \citep{March1991a}.

	When the literature discusses adaptive intelligence, it will focus--naturally--on examples of companies and industries that have adapted. It is easy to come up with examples of industries that have not adapted but have not been "creatively destroyed". Early electric vehicles were already mentioned above. Streetcar operators in the US have not adapted and have been artificially eradicated \citep{Norton2011}. Much of the traditional garment industry has disappeared in favor of mass products. The extraction industry has not adapted, but in many cases used political strategies to survive (call that adaption if you like).

	Let's be clear that rejecting creative destruction as a vehicle to regulate human development is not the same as saying that creative destruction cannot or does not happen. It is just to say that creative destruction should not be seen as the \textit{modus operandi} of history. It is a call for a more encompassing view that does not jump to a concept such as creative solution as an explanation. Returning to the notion of adaptive intelligence, we can now differentiate between creative destruction as a go-to response and other, more complex explanations. The downfall of newspapers and magazines--on paper and on the web--for instance could be attributed to creative destruction through social media sites such as Facebook. But while the aspect of destruction is certainly present--is something inferior really being replaced by something superior?

	Let's return to the example of a "new" electric vehicle industry. It would be tempting to look at Tesla becoming the most valuable car manufacturer and to conclude that creative destruction is at work again. The more encompassing view takes social processes into consideration. To just name three, there is (1) a repeated push by multiple national governments for new electric vehicles on the road. (2) The development of the lithium-ion battery mostly by researchers at public universities. (3) The preceding multiple "killings" of the electric vehicle industry, such as GM's deliberate discontinuation of the EV1 in 1999 subsequent to GM having a court strike down California's zero-emission vehicle mandate. The more encompassing view would also note that the 2010s may have be the decade of the emergence of the electric vehicle industry--this emergence however has been overshadowed by the victory march of the pickup truck, the increased emissions of which outweigh the aggregate environmental benefits of electric vehicles or increased efficiency of other passenger cars.\footnote{\url{https://www.nytimes.com/interactive/2019/10/10/climate/driving-emissions-map.html}, accessed 2020-03-16.}

	An important sidenote to our observations so far is the continued centrality of adaptive intelligence to learning in the theoretical approach laid out here. Collective learning may also take place where there is no necessity for it, just as we may regress. But we necessarily have observed learning at a collective level so far with regard to global challenges. E.g., with regard to the risks of nuclear weapons, the devastation stemming from global wars, the toxicity of chemicals, and the ozone hole--if we had not adapted, you, reader, would not be reading this right now. Of course it is also thinkable that one day there is a global challenge we do not adapt to through collective learning, maybe because adaptive learning is not a sufficient response.

	\section*{Reliability and Validity}

	In the main, we distinguish between two attributes of organizational and collective learning--reliability and validity. These concepts allow us to understand and analyze the complexity and contradictions that are apparent in organizational and collective learning in a world where futuristic technologies such as mobile phones, and collective failures such as the (lack of a) response to Covid and climate change in the US coexist. It's a pivot akin do \citep{Levinthal1993} or \citep{March1991a}, and I believe it could work out quite well if pulled off correctly. As an addition, we can go deeper into reliability and validity to introduce some processes.
	
	\clearpage
	\printbibliography

\end{document}