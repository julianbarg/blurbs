%!TEX root = ../main.tex

\subsection*{Validity}

% Correct predictions for the time being. Confusingly, correct predictions can occur even if the model is technically wrong--such processes are usually only discovered if later wrong predictions occur and a new valid model is created to rectify problems with the old model, that may have served well for a long time \citep{Madsen2010}.

Validity is the extent to which a learning process generates knowledge through which a collective can understand, predict, and control current and future events \citep{March1991a,Rerup2021}. The most illustrative example are explicit quantitative models that predict or allow to calculate specific values. 
% For instance, if a supermarket predicts \$19,000-\$21,000 in sales for toilet paper, and the actual sales end up being \$19,500, certainly we can say that the team went through a valid process to construct their confidence interval.
% \todo{Use "valid knowledge" here? Or continue to use "valid" and "reliable" only to describe processes?} 
% On the other hand, if then a global pandemic hits and a severe shortage in toilet paper occurs on a national level, we could then say that an invalid model of demand was constructed. That process would not be located at the level of the supermarket anymore, but the supermarket team's learning would have been invalidated.
For instance, \citeauthor{Manabe1967} predicted in \citeyear{Manabe1967} that a doubling of carbon dioxide in the atmosphere would lead to a temperature increase of 2.3°C--a value that was later validated by observations \citep{Forster2017}. Their efforts--in conjunction with the work of predecessors and research lab at Princeton--undoubtedly represents an example of valid organizational learning.

\begin{itemize}
	\item Unreliable learning
		\subitem \citet{Guarino2020}
		\subitem Predictions of sea-ice loss to conservative
\end{itemize}