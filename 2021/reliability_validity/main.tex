\documentclass{article}

\usepackage[natbibapa]{apacite}
\usepackage{hyperref}
\usepackage{todonotes}

\title{Reliability and Validity}
\author{
	Barg, Julian\\
	\texttt{jbarg.phd@ivey.ca}
	\and
	Zbaracki, Mark\\
	\texttt{mzbaracki@ivey.ca}
}

\begin{document}
	\maketitle

	%!TEX root = ../main.tex

\section*{Introduction}\label{sec:intro}

In 2004, Hurricane Ivan hit the Louisiana coast, and caused massive damages to the offshore oil industry. Most notably, an oil platform at the Mississippi River Delta was knocked over. Since then, the Taylor Energy oil platform has been continuously spilling oil into the Gulf of Mexico. At an estimated spill volume of 1.5-3.5 million barrels as of 2018, the spill rivals the 4 million barrels that Deepwater Horizon spilled into the Gulf in 2010.\footnote{\url{https://www.washingtonpost.com/energy-environment/2018/11/20/coast-guard-orders-cleanup-massive-year-oil-spill-gulf-mexico/}} Originally, after plugging some of the wells below the platform, Taylor Energy estimated the platform to only spill about two gallons of oil into the Gulf per day. However, in the aftermath of the Deepwater Horizon oil spill, Healthy Gulf--an NGO--discovered that the oil spill was much larger in scale. Aerial photos and videos show an oil slick a few miles long. Eventually, an independent contractor installed a containment system that collects about 1,000 gallons of oil per day.\footnote{\url{https://youtu.be/ztT45A501Tc}\label{foot:vice}}

For Taylor Energy, the original communication strategy held a lot of promise. After Hurricane Ivan toppled over the platform, Taylor Energy ceased to exist as an energy company. Only one permanent employee remained. However, Taylor Energy had been obliged to create a \$400 million fund for the clean-up cost--money that could potentially be returned to investors. As of 2021, Taylor Energy is suing the coast guard, the federal government, and the Couvillion Group--which contained the spill--for return of the money. Taylor Energy's efforts to close the case were not haphazard. In 2013, the organization convinced the participants of a joint workshop between three government agencies that capping the spill would likely result in more environmental damages than leaving the spill be \citep{Staves2013}. And in 2018, a PhD graduate of MIT issued a report stating that the oil sheen could conceivably have resulted from previously spilled oil being released by sediment, rather than actively spilling from the wells \citep{Camilli2018}.\textsuperscript{\ref{foot:vice}}

Taylor Energy's strategy is brazen, to put it mildly. There is a large disconnect between the large oil slick and the estimated spill volume of two gallons per day. Taylor Energy is suing the very company that captured video footage of the oil spill at the bottom of the ocean. In the words of Timmy Couvillion, CEO of the Couvillion Group "It's a joke".\textsuperscript{\ref{foot:vice}}

% Greenwashing describes the strategic use of misinformation \citep{Delmas2011}, or half-truths \citep{Lyon2011} for reputational gains in the environmental arena \citep{Bansal2004,Lyon2011}. 

	%!TEX root = ../main.tex

\subsection*{Scale}

% Mention cosmology episodes

Reliability and validity are applicable concepts for collective learning at any scale--be it at the level of a small organization or at the scale of our global society. On the same issue, one might observe a difference in validity and reliability of the learning process depending on the scale. For instance, the same discovery will have a different impact on collective knowledge at the level of the research lab where the discovery has taken place, at the level of the university, and at the national level.

\begin{itemize}
	\item The larger the organization (or collective) the more obvious the divergences. Most obvious at societal and global level.
	\item Managing divergences is a core feature of large organizations, such as corporations--think corporate culture. Employees might be quite cynical about the strategy of a corporation, but impression management may prevent that information to get out.
		\subitem Here, could mention that examples from sustainability are used--or could just move on and let the reader figure that out herself. 
		\subitem If wanted to make it explicit, reasoning would be that there is less need to disentangle truth from politics of truth--although there is still some.
\end{itemize}

	%!TEX root = ../main.tex

\subsection*{Validity}

% Correct predictions for the time being. Confusingly, correct predictions can occur even if the model is technically wrong--such processes are usually only discovered if later wrong predictions occur and a new valid model is created to rectify problems with the old model, that may have served well for a long time \citep{Madsen2010}.

Validity is the extent to which a learning process generates knowledge through which a collective can understand, predict, and control current and future events \citep{March1991a,Rerup2021}. The most illustrative example are explicit quantitative models that predict or allow to calculate specific values. 
% For instance, if a supermarket predicts \$19,000-\$21,000 in sales for toilet paper, and the actual sales end up being \$19,500, certainly we can say that the team went through a valid process to construct their confidence interval.
% \todo{Use "valid knowledge" here? Or continue to use "valid" and "reliable" only to describe processes?} 
% On the other hand, if then a global pandemic hits and a severe shortage in toilet paper occurs on a national level, we could then say that an invalid model of demand was constructed. That process would not be located at the level of the supermarket anymore, but the supermarket team's learning would have been invalidated.
For instance, \citeauthor{Manabe1967} predicted in \citeyear{Manabe1967} that a doubling of carbon dioxide in the atmosphere would lead to a temperature increase of 2.3°C--a value that was later validated by observations \citep{Forster2017}. Their efforts--in conjunction with the work of predecessors and research lab at Princeton--undoubtedly represents an example of valid organizational learning.

\begin{itemize}
	\item Unreliable learning
		\subitem \citet{Guarino2020}
		\subitem Predictions of sea-ice loss to conservative
\end{itemize}

	%!TEX root = ../main.tex

\subsection*{Reliability}

\begin{itemize}
	\item Reliable learning
	\item Unreliable learning
\end{itemize}

	%!TEX root = ../main.tex

\subsection*{Quadrants}

Similarly populate this section with examples, without drawing attention away from purpose of introducing terminology.

\begin{itemize}
	\item Expert/Technicist learning
		\subitem Low reliability \& high validity
	\item Popular/Populist learning
		\subitem High reliability \& low validity
	\item Regulatory/Political learning
		\subitem High reliability \& high validity
		\subitem Building coalitions around (selected) pieces of valid learning
	\item Skeptical learning
		\subitem Low validity \& low reliability
\end{itemize}

	%!TEX root = ../main.tex

\subsection*{Discussion}

\begin{itemize}
	\item Some observations of (political) processes that complicate processes of reliable and valid learning. Below some examples.
	\item \citet{Pailler2018}:
		\subitem Acting against better knowledge when (individual) political interests involved
	\item \citet{Aronczyk2019}:
		\subitem Interest groups insert themselves into learning processes early on, with complex implications for knowledge creation
		\subitem In this case, the interest groups has engaged in valid learning on stakeholders (MNCs) and their interests
			\subsubitem Or reliable learning, because their prediction on industry action did not pan out? Very political process at hand.
		\subitem Lessens primacy of validity with regard to \textit{physical} processes--the \textit{social} sphere gets to be considered
	\item \citet{Boudet2020}:
		\subitem "Resilience" of reliable over valid learning. Reliable learning can persevere in the face of overwhelming evidence that \textit{should} constitute cosmology episode
\end{itemize}

	% \section*{Reliability and validity in organizational learning}

	% This section provides a brief literature review of the existing work on reliability and validity.

	% \section*{Methods}

	% This section introduces which literature was used to develop the ideas that are introduced subsequently.

	% \section*{Findings}

	% This section works twofold. First, it lays out the processes of (un)reliable and (in)valid learning in clear terms. Then, the material from the methods section are used to show the processes in action and develop additional concepts. For instance, provide examples of invalid but reliable learning on a societal or organizational scale and show how the situation came to be. This section will be developed through additional tables on the literature that was introduced in the methods section. I have previously developed some of these tables, but have sighted additional literature with a critical view that now needs to be integrated (see \url{http://wiki.jbarg.net/Constructivism%20%26%20Ecology}).

	\section*{Conclusion}

	Just a summary? Spelling out implications for sustainability seems construed? Implications for different fields maybe?

	\clearpage
	\bibliography{bibliography}
	\bibliographystyle{apacite}

\end{document}