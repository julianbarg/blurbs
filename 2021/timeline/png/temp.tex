\documentclass[convert, 12pt]{standalone} 

\usepackage{times}

\usepackage{booktabs}
\usepackage{tabularx}

\begin{document}

	\begin{tabularx}{\textwidth}{r @{\hspace{0.5\tabcolsep}} l |@{\hspace{-2.3pt}$\bullet$ \hspace{5pt}} X}
		\toprule

		1874 & & DDT discovered.\\

		1939 & & DDT insect-killing properties are discovered.\\

		1941-1945 & & DDT used to protect US troops from insect-borne diseases in WW2.\\

		1945 & & The production of DDT for the military, now obsolete, is employed to benefit domestic agriculture.\\

		1947 & & Federal Insecticide, Fungicide, and Rodenticide Act is passed. Requires manufacturers to test and report product safety to government.\\

		1952 & & The question of the safety of DDT is touched on in \textit{Nature}, authors state that no ill-effects have been reported yet \citep{Davidson1952}.\\
 
		1959 & & Production of DDT peaks. DDT is the top-selling insecticide. At the same time, the literature has already picked up risks and exposure to DDT.\\

		1962 & & Rachel Carson publishes \textit{Silent Spring}.\\

		1963 & & \textit{Nature} and \textit{Ecology} review \textit{Silent Spring}.\\

		1963 & & The President' Science Advisory Committee (PSAC) calls for an end of the use of DDT and other pesticides.\\

		1968 & & Wisconsin classifies DDT as a water pollutant.\\

		1939 & & Various states implement bans of different uses of DDT.\\

		1971 & & A court orders the EPA to hold hearings on DDT.\\

		1972 & & The EPA bans DDT nationwide.\\

		1972 & & DDT usage has already declined by 67\% relative to 1962 levels.\\

		1973 & & The ban comes into effect.\\

	\end{tabularx}

\end{document}