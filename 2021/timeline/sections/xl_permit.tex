%!TEX root = ../main.tex

\begin{table}[H]
	\caption{Permit application process}

	\begin{tabularx}{\textwidth}{r @{\hspace{0.5\tabcolsep}} l |@{\timeline} X}
		\toprule

		2008 & June & TransCanada proposes the Keystone XL Pipeline.\parnote{\url{https://www.rigzone.com/news/oil_gas/a/149226/TransCanada_Proposes_Second_Oil_Pipeline/}}\\

		 & Sep & TransCanada submits application for Presidential permit to US Department of State\citep{Vann2012}.\\

		 & Dec & TransCanada submits application to Montana Department of Environmental Quality \citep{TransCanadaKeystone2008}.\\

		2009 & Mar & TransCanada submits application to South Dakota Public Utilities Commision \citep{TransCanadaKeystone2009}.\\

		2011 & Feb & Barack Obama vetoes a bill by the the Congress that would have approved the construction of Keystone XL.\parnote{\url{https://www.nbcnews.com/politics/white-house/obama-vetoes-keystone-xl-pipeline-bill-n311671}}\\

		 & Nov & Nebraska passes state laws LB1 and LB4, for the first time creating a formal application process for new pipelines, and providing funding to the Department of Environmental Quality (DEQ) to conduct impact studies on new pipelines.\parnote{\url{http://update.legislature.ne.gov/?p=5458}}\\

		 & Dec & Congress passes legislation to force the Department of State (DOS) to conclude its national interest determination for Keystone XL within 60 days \citep{USDOS2012}.\\

		2012 & Jan & The DOS rejects TransCanada's application on the grounds that it cannot determine whether the project is of national interest within the provided timeframe \citep{USDOS2012}.\\

		 & May & TransCanada reapplies for a Presidential permit for the construction of Keystone XL \citep{DoS2015}.\\

		2015 & Nov & The DoS and President Obama reject the application for a Presidential permit on the grounds that it would undermine US foreign policy with regard to climate change \citep{DoS2015,WhiteHouse2015}.\\

		2017 & Mar & The Trump DoS issues a Presidential permit, based on an unchanged analysis that now states in the Basis for Decision section that since 2015 "there have been numerous developments related to global action to address climate change, including announcements by many countries of their plans to do so" \citep[p. 29]{DoS2017}. 
		% As of 2020, there were 5 significant spills, see Table \ref{table:keystone}.

	\end{tabularx}
\parnotes
\end{table}

\begin{landscape}
\begin{table}[H]
	\caption{Representative Quotes on Permit Application Process}

	\begin{tabularx}{\linewidth}{>{\raggedright\hsize=.15\hsize}X X}
		\toprule

		TransCanada & 
		% "Based on the available information, the study produced a conservative incident frequency of 0.000119 incident per mile per 2 year. For any 1-mile segment, this probability is equivalent to one spill every 8,400 years [...]. 
		"Analysis of the current PHMSA dataset (2002 to present) indicates that the vast majority of actual pipeline spills are relatively small, with 50 percent of the spills consisting of 3.0 barrels or less. In 85 percent of the cases, the spill volume was 100 barrels or less, and less than 1,000 barrels in over 95 percent of the time. Oil spills of 10,000 barrels or greater only occurred in 0.5 percent of cases. These data demonstrate that most pipeline spills are small and very large releases of 10,000 barrels or more are extremely uncommon" \citep[p. 1-34]{TransCanadaKeystone2008}.\\\\

		TransCanada & "Pipeline incidents are infrequent and if a spill occurred, the volume would likely be 3 barrels or less. Keystone would initiate its ERP and emergency response teams would contain and clean up the spill. Appropriate remedial measures would be implemented to meet federal and state standards, which are protective of soils and their associated land uses"--part of TransCanada response regarding potential impact on soils, water, vegetation, wildlife, cultural resources, and public health and safety \citep[pp. 30, 31, 32, 33, 34, and 35]{TransCanadaKeystone2009}.\\\\

		Montana Department of Environmental Quality (DEQ) & "[A]dverse effects to [an animal species] are unlikely due to: 1) the low probability of a spill, 2) the low probability of a spill in a river reach where pallid sturgeon are present, and 3) the low probability of the spill reaching a river with pallid sturgeon in sufficient amounts to cause toxic effects" \citep{MontanaDEQ2012}.\\\\
	
		Obama administration & "The potential impacts  from  a  large  spill  would  be  similar  to  the  impacts  from  the  medium-sized  spill,  but  on  a  much  larger  scale" \citep[ES-18]{DoS2014}\\\\

		Industry-affiliated think tank & "[For the current] boom to continue, and for energy security to become a reality, the pipeline infrastructure to support North American oil and gas production must be expanded" \citep{Weinstein2014}
	\end{tabularx}

\end{table}
\end{landscape}