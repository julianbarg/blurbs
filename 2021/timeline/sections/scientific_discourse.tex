%!TEX root = ../main.tex

\begin{landscape}
\begin{table}
	\caption{Dominant paradigm on the long-term impacts of large oil spills}
	\begin{xltabular}{\linewidth}{r c c c X}
		\toprule

		Article/year & \tabincell{c}{Exxon \\Valdez} & \tabincell{c}{Deepwater \\Horizon} & \tabincell{c}{Kalamazoo \\spill} & Findings\\

		\midrule

		% 2010 & -- & -- & -- & Kalamazoo River pipeline spill occurs.\\

		\tabincell{r}{\citeauthor{Atlas2011} \\(\citeyear{Atlas2011})} & Yes & Yes & No & Oil-degrading microorganisms are ubiquitous in nature since hydrocarbons occur naturally. Full bioremediation can be accomplished by dispersing larger slicks, allowing microorganisms to break them down.\\

		\tabincell{r}{\citeauthor{Almeda2013} \\(\citeyear{Almeda2013})} & No & \tabincell{c}{Simulated \\in lab} & No & Some species, such as the group of zooplanktons, are particularly vulnerable to oil spills. Here, petroleum hydrocarbons might accumulate, especially when dispersants are also present. Overall, the long-term effect of oil spills still needs research.\\

		\tabincell{r}{\citeauthor{Fitzpatrick2015} \\(\citeyear{Fitzpatrick2015})\parnote{Government report}} & Briefly & Briefly & Yes & After oil spills in marine environment, oil binds to particles (e.g., mineral sediment, organic matter), forming oil-particle aggregates (OPAs). Submerged OPAs accumulate in sediments. They pose chronic risks to sediment-dwelling organisms and fish eggs, or can resurface following disturbance of the environment. Mechanical removal of OPAs is highly damaging.\\

		\tabincell{r}{\citeauthor{Barron2020} \\(\citeyear{Barron2020})} & Yes & Yes & No & Owing to complex interactions, the long-term impacts of large oil spills are more persistent and less predictable than previously assumed.\\

	\end{xltabular}

	\parnotes
\end{table}
\end{landscape}