%!TEX root = ../main.tex

\subsection*{Safety discourse}

\begin{table}[H]
	\caption{Keystone XL safety discourse}

	\begin{tabularx}{\textwidth}{r @{\hspace{0.5\tabcolsep}} l |@{\timeline} X}
		\toprule

		2006 & May & DNV consulting issues a confidential environmental impact report for TransCanada stating that Keystone can detect a large spill (50\% leak rate) within 9 minutes, and spill as small as 1.5\% leak rate within 138 minutes \citep{DNVConsulting2006}.\\

		2011 & July & Dr. John Stansbury of the University of Nebraska Omaha publishes an analysis that challenges the 2006 Keystone environmental impact report. For Keystone XL, Stansbury predicts 91 significant spills over 50 years, compared to 11 predicted by the DNV calculations \citep{Stansbury2011}.\\

	\end{tabularx}
	\parnotes
\end{table}

\begin{table}[H]
	\caption{Representative quotes on safety discourse}

	\begin{tabularx}{\linewidth}{>{\raggedright\hsize=.2\hsize}X X}
		\toprule

		DNV Consulting & "Overall, the likelihood of a leak greater than 50 barrels anywhere along the pipeline is estimated to be about 0.14 per year, or once every 7 years"--\citeauthor{DNVConsulting2006} on the expected safety performance of phase 1 and 2 of Keystone (\citeyear[p. 23]{DNVConsulting2006}).\\

		Dr. John Stansbury, University of Nebraska Omaha & "TransCanada made several assumptions that are highly questionable[...].
		% (1) TransCanada ignored historical data that represents 23 percent of historical pipeline spills, and (2) TransCanada assumed that its pipeline would be constructed so well that it would have only half as many spills as the other pipelines in service[...], even though they will operate the pipeline at higher temperatures and pressures and the crude oil that will be transported through the Keystone XL pipeline will be more corrosive than the conventional crude oil transported in existing pipelines.
		All of these factors tend to increase spill frequency; therefore, a more realistic assessment of expected frequency of significant spills is 0.00109 spills per year per mile [...] resulting in 91 major spills over a 50 year design life of the pipeline" \citep{Stansbury2011}.

	\end{tabularx}

\end{table}
