%!TEX root = ../main.tex

\subsubsection*{Environmental Management}

% If impact can be measured in citations, \citet{Hart1995b} should certainly be considered the most influential work cited here.
% \citet{Hart1995b} argues that while resources as defined by \citet{Wernerfelt1984} are important, only an intact natural environment can ensure that a firm can continue to operate.
\citet{Hart1995b} delivered what is possibly the first work in the business space to pick up the ideas of the WCED \citep{Montiel2014}. He introduces these ideas to a wider audience by building on the resource-based view. Hart identifies the natural environment as a potential "bottleneck" that could strangle corporations' long-term business. But alas, through reduction of their environmental emissions, and by low-impact development of the global South,\footnote{This development of the global South is the \textit{development} part in sustainable development.} the negative relationship between business and environment can be "severed" \citep[p. 996]{Hart1995b}, while regulations will only play a temporary role or follow from corporate action \citep[footnote p. 991, p. 995]{Hart1995b}.
% \footnote{Although most of his statements are qualified by the verbs "can" or "could".}
Hart revisited the topic in 2003, when he lays out more clearly the role of businesses \citep{Hart2003}. Sustainable development is described as a combination of (successful) industrialization, civic engagement, environmentally friendly technologies, and addressing population growth. Again, firms are actors in the spotlight that can find answers to all challenges and create value for themselves and others.

The next landmark article in this discourse is \citet{Bansal2005}, which was preceded by \citet{Bansal2002}. 
% Maybe because it targets managers alongside an academic audience, \citet{Bansal2002} lays out the vision of environmental management most explicitly. 
\citet{Bansal2002} mentions explicitly that sustainable development \textit{must} be implemented by one specific \textit{actor}: \textit{firms} \citep[p. 124]{Bansal2002}. 
% Central to the concept of sustainable development are the notion of an equity that is maintained indefinitely \citep[p. 123]{Bansal2002}.
% Environmental problems such as climate change stem from a divergence in understanding between societal and business actors. Business actors define sustainability inappropriately narrow and attend only to economic sustainability. However, 
The article lays out that economic, social, and environmental sustainability are interdependent, yet, managers might fail to implement social and environmental sustainability, because the rewards in those areas are diffuse. If firms are not to implement sustainable development because they don't understand the benefit to society, it might require for front runners to institutionalize measures first within their respective industries to get the ball rolling.
\citet{Bansal2005} reiterates the theme of interdependence between social, environmental, and economic sustainability, and emphasizes that the three are not at odds, but internally consistent \citep[. 197]{Bansal2005}. Through \textit{environmental management} firms can improve efficiencies and reduce impacts, ultimately achieving a better environmental performance and developing superior capabilities and resources. A further dissemination of sustainable development occurs based on "[r]esource based rationales" (\citealp[p. 200]{Bansal2005}, referring to \citealp{Barney1991a}) and institutional processes à la \citet{Meyer1977}.

\begin{itemize}
	\item \citet{Haugh2010}
		\subitem Moving agency to the employee-level, to strike a balance between \citet{Friedman1962} and \citet{Freeman1984}.
	\item \citet{Hernandez2012}
		\subitem Advancing notion of employee-level efforts toward sustainability. Explicitly mentions agency theory, but adds concept of stewardship as driver.
	\item \citet{Bansal2014} + \citet{Slawinski2015}
		\subitem Adding temporality to previous work as an example of bounded rationality.
	\item \citet{Hahn2014} 
		\subitem On second read, an outrageous article, not sure if I can tease that out.
	\item \citet{Heikkurinen2019}
		\subitem Theorizes specifically around agency.
\end{itemize}

% Boundedly rational \citep{Bansal2014}.

% "The field of business sustainability applies the principles of sustainable development to the firm level of analysis" \citep[in reference to Bansal, 2005]{Slawinski2015}.