%!TEX root = ../main.tex

What is sustainable? What is sustainability? There are some formulaic definitions that are widely accepted, including the dogmatic definition that "[s]ustainable development is development that meets the needs of the present without compromising the ability of future generations to meet their own needs" \citep[IV]{WCED1987}. But if you ask two sustainability scholars to define sustainability in their own terms, chances are you are still going to get very different answers. Potentially including the ubiquitous phrase "I know it when I see it" \citep{White2013}. All that is not to criticize sustainability scholarship. It is the nature of the beast of dealing with complex systems--such as our planet--that things are difficult to nail down.

Definitions attain importance in research because they shape the research that takes place after the introduction section. Things may be assumed away, and a definition can determine the direction of a research project. The definition of sustainable development by the World Commission on Environment and Development (WCED)--despite being the most widely used definition of sustainability by far--is still contested. On the one hand, the WCED definition is widely used as a working definition. On the other hand, the WCED definition is criticized for putting too much focus on economic \textit{development} for resolving current and future distributive and environmental problems \citep[e.g.,][6f.]{Constanza2014a}.

This blurb introduces the discourse on the WCED definition and what assumptions it introduces to research. It covers on the one hand research in business journals that apply the definition. These articles have in common that they turn to businesses and their policy decisions for potential solutions to common environmental or social problems. On the other hand, there is a critical 
% \todo{Avoid label "critical" for its connotations?} 
discourse on sustainable development that explicitly discusses the inherent assumptions and typically rejects the notions that companies are suitable agents that can be relied on for change \citep[e.g.,][]{Banerjee2003}.
% This critical  discourse maintains that action by businesses is and will remain insufficient, and our current inaction outside the business realm will be costly in environmental and economic terms.

The discourse is driven by competing assumptions on the ability for corporations to act as agentic actors for change. 
\begin{itemize}
	\item \hl{Add "third way" implications similar to \citet{Giddens1979}}.	
	\item Actually in some sense, this directly parallels Gidden's comments on structure vs. agency--e.g., \citet{Heikkurinen2019}.
	\item the "third way" is actually the extant literature I have extracted before--\url{http://wiki.jbarg.net/Constructivism%20%26%20Ecology.md#environmental-science-stream}
\end{itemize}
