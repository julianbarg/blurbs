\subsection*{Terminology and Definition}

A range of terms have emerged to describe the different takes on sustainability. For the critical stream, there is ecocentrism \citep{Purser1995}, deep ecology \citep{Newton2002}, the New Ecological Paradigm (\citealp{Catton1980} via \citealp{Hoffman2015}), critical epistemologies \citep{Ergene2020} or even ecofascism \citep{Newton2002}. For the "non-critical"\footnote{For the lack of a better word. It should be acknowledged that these authors themselves are also critical of an orthodox works such as \citet{Friedman1962}.} stream there are environmental management \citep{Purser1995}, corporate sustainability \citep{Hahn2014}, eco-modernism \citep{Springett2003}, and managerial epistemologies \citep{Ergene2020}. For the purpose of this work, I will adopt the terms ecocentrism and environmental management respectively \citep{Purser1995}.

% Note that in these citations, the labels represent attribution by the critical camp. Mainstream work also picks up terms such as "corporate sustainability", albeit "uncritical"--not as labels to describe themselves or differentiate a mainstream take from the critical take.

% A good benchmark is the WCED definition of sustainable development.

% An authors use of the WCED definition of sustainability is more useful as a benchmark, as it represents a deliberate choice by the author that differentiates the two camps. The mainstream work on sustainability routinely uses the definition and adopt the related language while the critical camp may pick it up to discuss its merits and shortcomings.