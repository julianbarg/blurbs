\documentclass{article}

\usepackage[natbib=true, style=apa, alldates=year]{biblatex}
\usepackage[pdfauthor={Julian Barg},pdftitle={Why to Venture into Gray Areas}]{hyperref}
\usepackage{nameref}

\addbibresource{bibliography.bib}

\title{Why to Venture into Gray Areas}

\author{
	Barg, Julian\\
	\texttt{jbarg.phd@ivey.ca}\\
	Ivey Business School
}

\sloppy

\begin{document}

	\maketitle
	\clearpage

	\begin{quote}
		"[E]very social actor knows a great deal about the conditions of reproduction of the society of which he or she is a member"

		--\citet{Giddens1979}, p. 5
	\end{quote}

	\section*{Introduction}

	\begin{itemize}
		\item Sustainability research in business journals has an unfortunate tendency to overestimate the agency of actors when positive impacts are concerned, and to underestimate their agency when concerning negative impacts \citep[cf.][]{Granovetter1985}.
		\item Example of heroic action, e.g., Bansal
		\item Example of structural or economic explanation, e.g., greenwashing
		\item Transition to \citet{Oliver1991}, agency of organization relative to institutional environment
	\end{itemize}

	\section*{The Pull of Gray Areas}

	\begin{itemize}
		\item Discus literature on decoupling, symbolic management, selective disclosure
		\item Invoke the notion that entering a gray area, such as greenwashing, can happen unintentionally, but it can also be the result of agentic action by some or all members of an organization.
	\end{itemize}

	Why would gray areas be attractive to organizations? Both in the literal and figurative sense--why would organizations be likely to end up in gray areas, and why would they like to end up in gray areas? There are three mechanisms that drive an organization toward a gray area. When an organizations goals are unambiguous, uncontested, and not uncontroversial, there is no reason why an organization should drift into a gray area. The more common arrangement is that organizations are to different degrees afflicted by goal conflicts, e.g., between departments \citep{Cyert1963}. However, when an organizations goals are ambiguous and contested, (1) a sub-coalition can move an organization into a gray area. Where that is the case, the gray area \textit{exists inside} the organization. For example, for Exxon to execute its legal strategy with regard to the \textit{Exxon Valdez} oil spill--which resulted in a reduction of the fine from \$5 billion to \$500 million at the expense of heightened emotional damages to the affected communities \citep{Kirsch2020}--only a sub-coalition of lawyers and executives would need to be "in the know". Most engineers and rig workers inside the company would likely be as unaware as the general public.

	Then there is (2) the case of goal conflicts that cannot be resolved. For instance, an organization may have multiple goals that each by themselves are not problematic, but in combination cannot be met. For instance, if a company has the goal to exploit resources and protect the environment while of course "maximizing shareholder value", and has a portfolio of oil extraction assets at the point of time in question, it is likely to meander into some kind of gray area unless one or more goals are abandoned--this is the space where decoupling is likely to occur, but here can also be a broader overlap with \textit{first scenario} here. Of course an organization can also (3) deliberately and consensually navigate into gray areas, where the organizational goal is not socially acceptable. For instance, organized crime does not need to be given the benefit of doubt.

	\begin{itemize}
		\item With regard to agency:
			\subitem In some cases, agency can be determined post-ex
			\subitem In many cases, agentic and structurally  action cannot be distinguished--and intentionality does not matter.
			\subitem In some cases, coalitions are a suitable explanation as to why organizations navigate into gray areas.
			\subitem But we need theories that also include nefarious motives by sub-coalitions and intent for bad actions, not only for good ones. 
	\end{itemize}

	\clearpage
	\section*{References}
	\printbibliography[heading=none]

\end{document}