\documentclass{article}

\usepackage{hyperref}
\usepackage{xcolor}
\usepackage{soul}
\usepackage[natbib=true, style=apa]{biblatex}

\title{Epistemic Communities and Collective Decision Making\\

{\large Or: \textit{How a decision is made in the modern world}}}
\author{
	Barg, Julian\\
	\texttt{jbarg.phd@ivey.ca}
}

\sethlcolor{pink}

\sloppy

\addbibresource{bibliography.bib}

% % Make is so only year is printed
% \AtEveryBibitem{
%   \clearfield{labelmonth}
%   \clearfield{labelday}
%   \clearfield{labelendmonth}
%   \clearfield{labelendday}
% }

\begin{document}
	\maketitle

	\newpage

	The management literature has made it abundantly clear that the rational model of decision making does not accurately describe how decisions are made in practice in organizations \citep{Eisenhardt1992}. The model and the research is mostly designed around organizations, or, more specifically, firms. The rational model of decision making has been applied more broadly. For example, policy making is also commonly thought of as a decision making process \citep{Allison1969}. Our models of decision making do apply to non-business situations. A special type of decision making occurs when the decision is made outside of the confines of an organization.

	Collectives do not generally behave like organizations. An organizations is a collective that has taken up a shared identity, with strong formal and informal decision making processes. This common umbrella makes easier many decision making processes \citep{Coase1937}. Other collectives usually have much a much weaker identity--think industry associations, the UN, or your local government \citep[cf.][]{March1995}. These organizations make decisions, too, and their decision making process is no less complex than that of organizations. If anything, the process is more complex, as widely diverging viewpoints clash.

	In the management literature, many of these collectives have been dealt with as "fields". That research generally focuses on gradual shifts of identity or general institutional processes. Collectives do however not always remain in that opaque form. Collectives have decision making processes and make decisions that matter--and are worthy of attention, see for instance \citet{Schussler2014}.

	\hl{[There is an opportunity for a very atmospheric intro. We go to a specific committee on a specific date, and observe how the specific decision is made on one of the (potential) marvels of modern engineering. Mention some of the technologies to be used. Here organizations and individuals come together to discuss the ins and outs and allow the experts of the commission to make a decision--then specify (sub-) decision to be made. Contrast that later with the qualitative data.]}

	\subsection*{Epistemic Communities}

	Where collectives makes decisions--if the organizational umbrella is missing--different epistemic communities meet. Sometimes they are more similar, then conflict might be limited to more detail questions. Sometimes they diverge more widely, and conflict is be more intense.

	In the absence of an organizational umbrella, conflict becomes more salient, as diverging identities and different viewpoints clash. The concept of epistemic communities \hl{[is great for understanding the nature of these differences, and the fundamental misfit in communication at work].}

	\subsection*{Keystone XL}

	The decision making process around Keystone XL excellently demonstrates \hl{[this]}. The section below introduces some epistemic communities, and some episodes that exemplify their epistemology at work. The examples are mostly taken from one specific event. \hl{Every five years, the Public Utilities Commission of South Dakota has to decide whether the permit for a pipeline that has not yet been constructed is still valid, or whether the conditions of the initial permit are not met anymore. The original permit has been issued in 2009, but TransCanada had not yet received the Presidential Permit (dependent on the National Interest Determination of the Department of State). So the Public Utilities Commission began the process (Order Accepting Certification) in 2014. The process became a battleground of environmental groups with TransCanada. Most of the hearings in question took place in 2015. In 2016, the order was finally granted. Challenges in front of the South Dakota Supreme Court were not settled until 2018. Much attention received a whistleblower, an engineer who had previously worked at TransCanada and testified before the Senate of Canada.}

	\subsubsection*{Procedural}

	In this specific case, the decision making process is hosted by an administrative unit of the government. Governmental decision making is characterized by a high degree of \hl{[formal rules]}. The South Dakota Public Utilities Commission exhibits this epistemology in its preference for a formal structure. \hl{[Exemplify organization of overall process. Show South Dakota PUC making decisions about structure in favor of TransCanada to follow formal rules. Trying to clearly delineate the scope of the process and not allow activists to let scope escalate.]} By diverging from formal rules, the administrator risks making a decision that will be contested. A negative decision would have undoubtedly lead to TransCanada going to court, where decisions would have had to be defended against fierce lawyers. At the same time, the administrator would need to make a decision that is impervious to activists, too. Not offering an opening to any side. These concerns were validated: four challenges to the final decision and process were brought before the South Dakota Supreme Court.\footnote{
		\url{https://puc.sd.gov/dockets/Civil/2017/supct28331.aspx}, 
		\url{https://puc.sd.gov/dockets/Civil/2017/supct28332.aspx}, 
		\url{https://puc.sd.gov/dockets/Civil/2017/supct28333.aspx}, 
		\url{https://puc.sd.gov/dockets/Civil/2017/supct28334.aspx}.
	}

	\hl{[The procedural epistemology at play is also the reason why despite the many questions raised by environmental interest groups, the testimonies are overall very short. The testimony of Dan King, Vice-President at TransCanada for instance is only about 2,000 words, certainly not enough to discuss and come to a final conclusion on in-depth questions regarding the reliability of TransCanada's safety precautions and regulations \mbox{\citep{King2015}}. Interestingly, there is something going on here that does not intuitively fit the procedural epistemology. Witnesses reveal their professional resume or their personal background, which seem to hold quite some importance in the decision making process. What Dan King says seems to hold some weight just by who is saying it--e.g, he is in charge of engineering and reliability, he has been with TransCanada for 32 years, has a lot of experience, and was the in charge of department of the whistleblower, which he mentions early in his testimony \mbox{\citep[][p. 2]{King2015}}.]}

	\subsubsection*{Managerial}

	The managerial mindset views collective decision making as a process that is to be navigated. TransCanada's managers provide clear, unambiguous answers that do not offer any surface for attack.

	\begin{quote}
		[Question] 7. Mr. Vokes suggests that a fatality occurred on the Gulf Coast Project caused by pipe falling off the skids. Is that accurate? 

		Answer: No, it is not. 

		[Moving on to next question.]

		\citep[p. 4]{King2015a}
	\end{quote}

	The responses can be clear, without offering any surface for attack, yet not represent the complete picture. A popular example is the overly specific official denial--the spokesperson mentions a specific narrative that has surfaced in the media, with specific details, and states that that narrative is not correct. However, a very similar narrative with slightly diverging details happens to be accurate.

	\begin{quote}
		[Question] 11. Mr. Vokes alleges a "salt induced microcracking" problem with pipe ordered for the Keystone XL pipeline. Can you comment on that allegation? 

		Answer: There is no phenomenon known as "salt induced microcracking" in the pipeline industry. [...]

		\citep[p. 7]{King2015}
	\end{quote}

	Mr. King is right.\footnote{\url{https://www.youtube.com/watch?v=hou0lU8WMgo}} "Salt induced microcracking" is not a term that is used in the engineering literature. However, the presence of salt induces corrosion, which leads to pipelines cracking under pressure.

	The managerial epistemology allows for much planning and strategic behavior. For instance, it had been clear to TransCanada that the whistleblower, Evan Vokes, could turn into a major problem for the company. Evan Vokes later used access to information to obtain this internal email from TransCanada regarding his case:\clearpage % To avoid weird page counting restart bug.

	\begin{quote}
		[Redacted] and [redacted] are managing the [Evan Vokes] credibility issue. My understanding is that we have been reasonable successful at influencing authorities [redacted] and pointing out [Evan Vokes] is disgruntled, and actually had the responsibility to correct these same matters and did not.\footnote{\url{http://s3.documentcloud.org/documents/1151198/transcanadaemailaboutevanvokesfeb2012-3.pdf}, see also \url{https://thenarwhal.ca/details-pipeline-safety-whistleblower-emerge-transcanada-keystone-xl-delay/}.}
	\end{quote}

	\hl{Another example: a common complain of the public (i.e., communal epistemology) is that TransCanada--a Canadian company!--is conducting this project. The complaint: the American environment is suffering, and it is not even an American company that is profiting. TransCanada in 2019 changed its name to TC Energy, explicitly to de-emphasize its Canadian identity, and to create a North American identity. Certainly, that resolves the problem of invoking a nationalistic mindset of members of the American public when they only just see the company name}\footnote{\url{https://www.cbc.ca/news/canada/calgary/trans-canada-name-change-tc-energy-1.4971068}.}

	\subsubsection*{Rational}

	\hl{[This section is on engineering. The belief that if something is engineered right, it is safe, as expressed for instance by the whistleblower Evan Voke. Similarly, the belief that if something is measured correctly, one can make definite conclusions, but that the wrong assumptions have been made. The battle in front of the committee is than a conflict to arrive at rational assumptions, in order to be able to make a statement. Expressed by \mbox{\citet{Stansbury2011}}.]}

	In an interview with the legal press, the whistleblower highlights the different between this point of view, and the point of view of justice-oriented activists. They are somewhat allies of convenience:

	\begin{quote}
		Vokes is something of a superstar among good government and environment groups in Canada. The Council of Canadians even gave him a whistleblower award. 

		"I have had support from left leaning organizations as long as I'm supporting their point of view. But my point of 	view is not the same as theirs." 

		How is your point of view different from reporters and the Council of Canadians — the group that gave you the whistleblower award? 

		"I believe that pipelines are an important part of our transportation network for fossil fuels," Vokes said.

		You support the fossil fuel industry and pipelines for moving fossil fuels?

		"In general yes." 

		Your criticism is that it is not being done safely? 

		"Correct." 

		\citep{CorporateCrimeReporter2017}
	\end{quote}

	\subsubsection*{Justice-oriented}

	\hl{[Justice-oriented things go here]}

	\subsubsection*{Communal}

	\hl{[This section is dedicated to the tired, the poor, the huddled masses yearning to breath free. The general public participates in hearings and submits comments--over 100,000 of them on Keystone XL to the federal government. Sometimes, these comments are a confused read, some passages borderline incoherent. Sometimes they show more care, but in general they are short, and the authors express worry and/or outrage. This impression is occasionally broken up by individual authors that worry about the oil and gas industry and express outrage that Keystone XL might not be built.

	In the hearings, this expression of concern and yearning for answers clashes with the procedural mindset. Many residents have worries about their property or their immediate environments. They come to the hearings and seek answers that directly relate to their life and their worries. However, most question are not permitted, because they fall outside the scope of what the hearing is about. In the case of the Keystone XL hearing in front of the South Dakota Public Utilities Commission (2014-2018) the question is strictly whether TransCanada still meets the requirements of the permit that was issued in 2009. Therefore, most questions of participants in the hearings are inadmissible from the get go. The process leaves the members of the public that participate end up sounding defeated.]}

	\subsection*{Tidbits}

	\hl{The whistleblower indicates another interesting phenomenon. Managers delegate to engineers to get the response they want. Regulators "delegate" to the company, represented by management. Maybe they are happy to make a decision that they don't like as long as they can point to the papers (e.g., in front of courts) and say "it's all here! Look!" The problem emerges when managers overrule engineers and engineers say "I will do what you want me to do, and say what you want me to say, but I document that you asked me to do it in case something blows up. 'Cover your ass'. Document everything." There is a long chain of accountability that hides away most problems for long enough that it becomes intractable. What's more, most things do not actually blow up, they merely \textit{could} blow up. Which may be mistaken for, but is not the same as, the system working:}

	\begin{quote}
		\textbf{Senator Massicotte:} Thank you, Mr. Vokes, for being with us. I also compliment you for coming and speaking out on your concerns.

		Senator Unger was basically on the same track. We hear you. I have to admit it is very detailed, probably beyond our scope to follow your reasoning, but we know you are sincere. Yet, we look at the history of incidents and accidents for TransCanada, and there is nothing alarming there.

		\citep{SenateofCanada2013} 
	\end{quote}

	\hl{And, a factor that should not be underestimated is that the regulator may feel that the failure of a company also indicates the failure of a regulator, when e.g., an activist indicts corporate misbehavior:}

	\begin{quote}
		\textbf{Senator Wallace:} To put it mildly, that is a very broad, sweeping indictment from top to bottom of all regulatory and industry participants in pipeline transportation. However, that is your opinion.

		\citep{SenateofCanada2013} 
	\end{quote}

	\hl{There is a similar disconnect between the managerial and engineering epistemology. From an engineering perspective, the pipeline technology is controllable, it just needs to be done right. }

	\begin{quote}
		\textbf{Mr. Vokes}: The first and most important thing is the key to making sure that the regulations and codes are carried out, which is inspectors. The United States is good with the concept of the OQ qualified operators. I do like that concept, where people are formally trained and accountable and allowed to stop the work regardless of how much of a schedule disaster it is. 

		If I have a safety violation where a person may become hurt, the National Energy Board regulations allow me to stop the work. If I have code violations, everyone says, "Don't worry. Get it done. Deal with it later." That is the wrong attitude. We see a lot of that.

		\cite{SenateofCanada2013}
	\end{quote}

	\hl{One can imagine how that in practice would lead to the kind of cascading delays that cause the public to perceive a project as unreliable and inefficient. An email from a manager to one of the whistleblowers colleagues in engineering shows what happens in practice, and the difference in epistemology between management and engineers within TransCanada:}

	\begin{quote}
	
		\textbf{From}: Russell Wong

		\textbf{Sent}: Tuesday, January 18, 2011 02:08 PM

		\textbf{To}: Matt Cetiner

		\textbf{Cc}: Evan Vokes; Suman Basak; 'dhodgkinson@telus.net' \textless dhodgkinson@telus.net\textgreater ; David Taylor; Darryl Sandquist; David Penning; Meera Kothari

		\textbf{Subject}: FW: NDE Contractors - Phase 3 and Phase 4\\

		Matt

		I spoke with Dave Hodgkinson today to get background information on Weldsonix. The start of using Weldsonix for TransCanada projects was back in 2004 for Peerless I project and the conclusion of their performance was poor. Since then TransCanada Keystone signed a contract with Weldsonix on March 1, 2008. Weldsonix was later disqualified probably around spring 2009 by Dave. H. I suspect a letter was never issued to Weldsonix notifying them of their disqualification. The reasons for disqualification of Weldsonix can be found in the Supplier Management System and according to Dave H. the reasons are still valid today.

		Dave H. who has had experience working with Weldsonix is advising TransCanada and Keystone never to use them again with the following reasons: 

		\begin{enumerate}
			\item Lack of management support for the project 
			\item Lack of technical support for the project
			\item High turnover of employees
			\item Lack of proper maintenance on equipment 
			\item Lack of QA/QC competency / commitment
			\item Operates the project at the least cost 
			\item Broken promises that they will perform better next time

		\end{enumerate}

		Based on the following points it is my opinion that Weldsonix poses high risk to the project from a supplier performance and pipeline integrity perspective. Due to shortages of AUT suppliers, we should proceed with pre-qualification of Oceaneering as recommended by Eng. Gov. Russell 

		 ---

		\textbf{From}: Meera Kothari

		\textbf{Sent}: Tuesday, January 18, 2011 2:12 PM

		\textbf{To}: Russell Wong; Matt Cetiner

		\textbf{Cc}: Evan Vokes; Suman Basak; 'dhodgkinson@telus.net'; David Taylor; Darryl Sandquist; David Penning

		\textbf{Subject}: Re: NDE Contractors - Phase 3 and Phase 4\\

		we are going to proceed with weldsonix on the list for phase 4 and they will continue as the nde contractor for michels phase 3. 

		We are not going to qualify oceaneering for any keystone related work. 

		Please stop these emails. This is a project engineering decision. 

		Thanks 

		Meera\footnote{\url{https://puc.sd.gov/commission/dockets/HydrocarbonPipeline/2014/HP14-001/draexhibits/121.pdf}, see also \url{https://insideclimatenews.org/news/01052014/did-transcanada-try-discredit-pipeline-safety-whistleblower/}.} 
	\end{quote}

	% Managerial, procedural, rational, justice-oriented

	% Allies of convenience, delegate

	% If you have to switch from one epistemology to another, that is tough. EV had to do that when he went from whistleblower to the senate to participant in South Dakote hearing

	% In the end, I don't believe that decisions made in organizations are all that different from the mode of decision making presented here. They are just usually more simple, because less people are involved, and hierarchies are usually more [strong?].

	% It is surprising that after so many questions have been raised, the testimonies are all so short.

	\clearpage
	\printbibliography

\end{document}