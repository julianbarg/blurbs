%!TEX root = ../main.tex

\subsection*{Methods}

The purpose of this study is to extract the meaning of words and phrases from text \citep{Parker1992}. We identify the main themes of the discourse by reviewing and coding the public forums. To decode their meaning, we first reference the timeline of events that was compiled in the first step in the section \nameref{sec:data} \citep{Maguire2009}. We then turn to the texts that the community has produced ahead of the forum. Connecting themes from the forums with texts that were produced ahead of the forum, allows us to more accurately decode their meanings. Texts that are specific to the forum sometimes reveal a community's strategy for the forum.

We have now obtained coded meanings and strategies on the forum. The next step is to compare coded themes before, during, and after. To obtain a process model of the discourse, we code the texts produced by all communities ahead and after each forum. Discrepancies between the before and during provide more information on the communities' strategies. Discrepancies between the during and after suggest an evolving understanding through exchange at the forum. Finally, we track themes across communities and across time throughout multiple forums. This step allows for the identification of more connections and overarching trends.

In addition to the analysis of texts, this analysis uses interview data. The primary function of the interviews is to test the robustness of our findings. For that purpose, I compile timelines on the discourse for each community. These timelines are shared with interviewees from the communities to triangulate events and ensure completeness of the data. Where data is incomplete, we return to the previous step and review the findings, before carrying out more interviews.
