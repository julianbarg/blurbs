%!TEX root = ../main.tex

\section*{Introduction}

% \begin{quote}
% 	\begin{flushleft}
% 	\textit{Ne ego si iterum eodem modo vicero, sine ullo milite Epirum revertar.}

% 	If I achieve such a victory again, I shall return to Epirus without any soldier.

% 	--Paulus Orosius (ca. 375-420 AD) in \textit{Historiarum Adversum Paganos Libri}
% 	\end{flushleft}
% \end{quote}

On January 20, 2021, the very day Joe Biden took office as the American president, he issued the "Exective Order on Protecting Public Health and the Environment and Restoring Science to Tackle the Climate Crisis". 
% \footnote{\url{https://www.whitehouse.gov/briefing-room/presidential-actions/2021/01/20/executive-order-protecting-public-health-and-environment-and-restoring-science-to-tackle-climate-crisis/}, accessed 2021-04-12.}
The order contains six broad policy decisions--and one very specific one: TransCanada's construction permit for Keystone XL is revoked. The \textit{Keystone XL} pipeline was supposed to be TransCanada's follow-up project to the \textit{Keystone} pipeline which transports diluted bitumen from Alberta, Canada to Texas. Construction began in April 2020, after the permit was issued in 2019. The order revoking TransCanada's permit refers to a 2015 review of Keystone XL by the Department of State which found the pipeline to have a negligible economic impact, while posing a significant threat to the environment \citep{DoS2015}.

On the surface, the revocation of TransCanada's permit looks like a late victory for the environment. \textit{Keystone} was built and Keystone XL began construction, but America finally came around. After an extensive discourse, the concern for the environment has come to dominate. For the oil and gas industry, the defeat of Keystone XL is highly symbolic and "may signal the end of major U.S. oil infrastructure" \citep{Freitas2021}. 
% \footnote{\url{https://www.worldoil.com/news/2021/1/20/keystone-xl-shutdown-may-signal-the-end-of-major-us-oil-infrastructure}, accessed 2021-05-14.} 
In other words, \textit{Keystone XL} represents an important step toward the deinstitutionalization of pipelines through discourse \citep[cf.][]{Maguire2009}. 

However, doubt should remain: How robust is the new status quo? Many actors continue to stand behind TransCanada. There are those who are directly or indirectly employed by the industry, and those studying subjects with deep ties to resource extraction, such as geology or engineering. And there are the communities for which oil \& gas offers livelihood and identity--in an age where more oil can be extracted by less hands, maybe more of the second than the first. 
% \footnote{Compare \url{www.statista.com/study/10938/us-oil-industry-statista-dossier/}, accessed 2021-05-15.}
The three states that Keystone XL was supposed to cross never withdrew support for the project--TransCanada was at all time able to obtain all necessary permits at the state level. On the national level, the Koch family foundations, the American Petroleum Institute, and individuals from the oil \& gas industry such as ExxonMobil CEO turned Secretary of State Rex Tillerson continue to hold a lot of sway and lobby for the interests of the industry.

My thesis studies the pipeline industry in the context of the cultural war that is being fought between environmentalists and proponents of the oil \& gas industry. The second study, my job market paper, focuses on Keystone XL. The cultural war has for the most part remained a war of words but it has a distinctly physical dimension that is best exemplified by the 2016 Battle at Standing Rock related to the Dakota Access Pipeline. 
% \footnote{\url{https://www.theatlantic.com/video/index/507728/solidarity-standing-rock/}, accessed 2021-05-15.}
Sustainability scholarship conventionally focuses on a "cleaner" vision of progress that follows either from "enlightened" individuals inside of organizations \citep[e.g.,][]{Howard-Grenville2017}, or from economic motives \citep[e.g.,][]{Flammer2015}. The oil \& gas industry holds an antithetical vision, where progress follows from economic development, and resource extraction is an economic necessity. Here, the OT literature can contribute with its rich body of works analyzing social process--think organizations and institutions, networks, and power \citep{Davis2015,Ergene2020}.

% oil \& gas stands in for a vision of modernity, who perceive resource extraction as an economic necessity.

% Oil \& gas also stands symbolically for a vision of modernity

% Two sides to the issue, not ontologically, but epistemologically.