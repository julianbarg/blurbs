%!TEX root = ../main.tex

\section*{Data Collection}\label{sec:data}

% Data collection for study two involves five steps. First, a timeline of events is created \citep{Maguire2009}. From that timeline, we select forums that facilitate a discourse \citep{Latour2005}. Then, texts from the participating communities on the discourse are gathered. The data analysis is carried out on the basis of these texts. Finally, I interview individual members of the communities to test the robustness of my fiends and ensure convergence.

This section introduces the data collection for study two. To understand the discourse on Keystone XL, and how it unfolded over time, we require a sample of texts that encompasses multiple relevant actors over an extended period of time. The first step toward building this corpus is to develop a timeline of events \citep{Maguire2009}. The basis for that timeline are newspaper articles from the sampling period of 2008-2021. These articles were obtained from five major American national news outlets
% The Wall Street Journal, The New York Times, Los Angeles Times, and The Washington Post 
through Factiva by searching for the Keywords "Keystone XL", and "KXL". Further, articles on Keystone XL over the sampling period were collected from the Oil \& Gas Journal. 
% An excerpt of the timeline can be seen in \autoref{tab:timeline}.

% %!TEX root = ../main.tex

\begin{table}
\caption{Keystone XL--Timeline of selected events}
\label{tab:timeline}
\begin{tabularx}{\textwidth}{r @{\hspace{0.5\tabcolsep}} l |@{\timeline} X}
% \begin{tabular}{r @{\hspace{0.5\tabcolsep}} l |@{\timeline} >{\raggedright\arraybackslash}p{9cm}}
\toprule

% 2005 & Feb 	 & TransCanada unveils plans to build the Keystone pipeline to transport diluted bitumen from Alberta, Canada to Texas.\\
2008 & Mar   & The Bush administration issues the Presidential Permit for the \textit{Keystone Pipeline}, predecessor of the \textit{Keystone XL} pipeline.\\
	 & June  & TransCanada proposes the Keystone XL pipeline.\\
2010 & April & The Deepwater Horizon oil platform explodes before spilling 4.9 million barrels of oil in the the Gulf of Mexico.\\
	 & June  & Phase 1 of Keystone begins delivering oil from Alberta, Canada to Steele City, Nebraska in the US.\\
	 & July  & An Enbridge pipeline spills over 20,000 barrels of diluted bitumen into the Kalamazoo River in Michigan.\\
2012 & Jan   & The Department of State rejects the Keystone XL application after congress passes legislation to require the completion of the National Interest Determination within 60 days.\\
2014 & Jan	 & The Keystone pipeline is completed and begins delivering oil from Canada to Texas.\\
2015 & Nov	 & The Department of State rejects the second Keystone XL application.\\
2016 & Jan 	 & In his first week in office, Trump signs a presidential memorandum that invites TransCanada to resubmit its application.\\
2017 & Mar   & Trump signs the Presidential Permit, based on a slightly revised National Interest Determination that now rules in favor of Keystone XL. The permit later faces challenges in court on procedural grounds.\\
2019 & Mar 	 & Trump revokes the permit for Keystone XL and issues a new permit. Legal challenges continue.\\
2020 & April & Construction starts on Keystone XL.\\
2021 & Jan 	 & Biden revokes the permit for Keystone XL.\\
\bottomrule

\end{tabularx}
% \end{tabular}
\end{table}

The timeline on Keystone XL includes multiple public forums \citep{Latour2005}. These public forums serve two functions in this study. (1) They allow for the identification of relevant participants in the discourse. 12 types of participants participate in the forums, 6 of which are specific organizations that participate repeatedly. 
% (see \autoref{tab:forums}). 
(2) The public forums are a compressed version of the discourse itself, where actors lay out their understanding of the permit application process, of Keystone XL, and of its environmental impact. The actors also respond to each others claims in this forum, laying bare the extent to which they have engaged with each others' arguments. I.e., a rebuttal testimony begins with a summary of the other side's position, and make a useful gauge of engagement and comprehension.

% %!TEX root = ../main.tex

\global\pdfpageattr\expandafter{\the\pdfpageattr/Rotate 90}

\begin{landscape}
\begin{table}[ht]
\tiny
\centering\caption{Discourse illustration}
\begin{tabular}{p{3.5cm}@{\hskip 5mm} r s s s s s s s s s s s s s s s}
	\toprule

	Event & Time 
		& \rota{Keystone} 
		& \rota{State government} 
		& \rota{Federal government} 
		& \rota{Local indigenous organization(s)} 
		& \rota{Local residents}
		& \rota{Labor unions}
		& \rota{Dakota Rural Action} 
		& \rota{Bold Nebraska}
		& \rota{Sierra Club} 
		& \rota{350.org}
		& \rota{Indigenous Environmental Network}
		& \rota{TransCanada whistleblower}
		\\

	\midrule

	Canada National Energy Board--OH-1-2009\parnote{\url{https://apps.cer-rec.gc.ca/REGDOCS/Item/View/550305}} & Mostly 2009 & 
		\ye & \no & \ye & \ye & \hm & \ye & \no & \no & \ye & \no & \no & \no\\

	South Dakota Public Utilities Commission--HP09-001\parnote{\url{https://puc.sd.gov/Dockets/HydrocarbonPipeline/2009/hp09-001.aspx}} & Mostly 2009 &
		\ye & \ye & \no & \no & \no & \no & \ye & \no & \no & \no & \no & \no\\

	Montana Department of Environmental Quality--Certificate of Compliance & March 2012\parnote{\url{https://deq.mt.gov/DEQAdmin/mfs/keystonexl/keystonexlcomprehensive}} &
		\ye & \ye & \no & \no & \no & \no & \no & \no & \no & \no & \no \\

	Senate of Canada--TransCanada whistleblower testimony & June 6, 2013 &
		\fu & \no & \ye & \no & \no & \no & \no & \no & \no & \no & \no & \ye\\

	South Dakota Public Utilities Commission--HP14-001\parnote{\url{https://puc.sd.gov/Dockets/HydrocarbonPipeline/2014/hp14-001.aspx}} & Mostly 2014 & 
		\ye & \ye & \no & \ye & \ye & \no & \ye & \ye & \ye & \ye & \ye & \ye\\

	Nebraska Public Service Commission--OP-0003\parnote{\url{https://psc.nebraska.gov/natural-gas/keystone-xl-pipeline}} & Mostly 2017 & 
		\ye & \ye & \no & \ye & \ye & \ye & \no & \ye & \ye & \no & \no\\

	Montana Department of Environmental Quality--401 Water Quality Certification & Mostly 2020 &
		\ye & \ye & \no & \no & \ye & \no & \no & \ye & \no & \no & \no\\

	\bottomrule
\end{tabular}
\parnotes
\end{table}
\end{landscape}

\global\pdfpageattr\expandafter{\the\pdfpageattr/Rotate 0}

By selecting on the participation in forums, we obtain a sample of actors that are both knowledgeable and engaged. To participate in the public forum, an organization needs to both familiarize itself with the formal process and strategize, as well as obtain and organize expert knowledge. 
% Some of the participants, such as the Sierra Club or Bold Nebraska are or become communications experts over the course of the discourse.

The final step toward creating a sample of texts is to gather the communication of the actors on Keystone XL over the time of the discourse. Most of the communities have formed and organize themselves through websites. Documents on TransCanada, Keystone XL, and the Sierra Club are also published online. The documents are downloaded by hand, or automatically through web scraping where appropriate. In some cases, the texts are automatically filtered to retain only those that discuss Keystone XL, using the keywords "Keystone XL" and "KXL". The filtered texts, and the texts from step two constitute the final sample that is imported into ATLAS.ti for the data analysis.

% Some of the forums were open to public participation. Comments by residents were also collected but were too numerous to manually sight. Topic modeling and sentiment analysis were used to identify substreams of that corpus. Samples from each substream then entered the analysis.