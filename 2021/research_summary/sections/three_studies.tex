%!TEX root = ../main.tex

\subsection*{Studies}

This section introduces three studies that are designed to generate research insights from the frail progress on the deinstitutionalization of oil and gas that is exemplified by the revocation of the permit for Keystone XL. The remainder of this document, after a brief summary of the \textit{quantitative} data I have collected on the industry, focuses on the \textit{qualitative} methods and data for the second study. The studies draw on institutional theory and organizational learning. Institutional theory is useful for explaining the stability of the current structure, and the challenges around substantial (as opposed to symbolic) change \citep{Meyer1977,Hoffman2015}. Organizational learning on the other hand emphasizes the potential for adaption, knowledge sharing, and vicarious observation, while keeping one eye on obstacles \citep{Madsen2010,Madsen2018,Rerup2021}.

Underlying all three studies is the understanding I have briefly developed above: that the polarization of the discourse on oil and gas has not been overcome by the defeat of Keystone XL. That the recent institutional developments in this area are characterized by twists and turns. And most importantly, that we should pay more attention to pipeline projects such as Keystone XL as potential precursor of the deinstitutionalization--or resurrection!-- of oil and gas, and "expect the unexpected".

The \textit{first study} focuses on the industry side of the discourse. The operation and construction of pipelines requires a social license. The US experiences over 100 significant
\footnote{The Pipeline and Hazardous Materials Safety Administration describes a pipeline spill as significant if the spill volume exceeds 50 barrels, or if one of three other criteria is met--see \url{https://www.phmsa.dot.gov/data-and-statistics/pipeline/pipeline-incident-flagged-files}, accessed 2021-05-13.} 
pipeline spills every year.
\footnote{See \url{https://www.jbarg.net/incident_dashboard.html}, accessed 2021-05-13.} Pipeline operators routinely make references to technology--both clean-up technology, and pipeline safety technology--in their statements on recent pipeline spills. Since both the quality of implementation and effectiveness of these new safety technologies is highly opaque for external audiences, the risk of decoupling and symbolic adaption is high \citep{Wijen2014a}. In other words, pipeline operators can use symbolic references to technology in conjunction with selective disclosure of information on spill causes as a means of greenwashing \citep{Lyon2015}. Study one explores the potential of technology as a device for greenwashing, using a mix of quantitative data on pipeline networks and spills, and qualitative descriptions of spill causes and responses.

The first study's theoretical focus on greenwashing limits the analysis to only one side of a richer discourse. Five communities have participated in the discourse on Keystone XL: pipeline operators, environmental activists, political decision makers, local residents, and the scientific community. The \textit{second study} opens up the blinds to provide the complete picture of this multivocal environment. Participants meet in public forums across multiple years \citep{Latour2005} and produce their own texts before and after, allowing us to repeatedly observe their beliefs, information exchange, and strategizing over time. \citet{Maguire2009} describe deinstitutionalization as a process of discourse and collective alteration of institutions. In the case of Keystone XL, we observe only a partial reconfiguration of knowledge networks, which leads to a frail deinstitutionalization of large oil and gas infrastructure projects \citep[cf.][]{Latour1984}. Study two studies focuses on the different communities that participate in the discourse on Keystone XL \citep{Aronczyk2019,KnorrCetina1999}, and how their participation on the Keystone XL discourse affects their understanding of the environmental risks and impacts related to pipelines and fossil fuels. Finally, I develop a model of the social process of deinstitutionalization in a multivocal environment.
% The ultimate rejection of the Keystone XL application neither resolves the conflicts nor does it lead to a stable new status quo. The conflict has given rise to new networks of pipeline opponents. But the sparring communities and their fundamental disagreements persists after the conflict. The temporary new status quo can be attributed to the new constellation that has emerged between the belligerents \citep{Latour1984}.

The \textit{third study} is a theory paper that applies the insights from studies one and two to the theory on organizational and collective learning. Whether a pipeline operators can maintain their public license does not only depend on the accuracy of the representations of their assets that they share with their audience. Rather, their models of the world need to become shared across a sufficient share of audience members. The learning literature similarly emphasizes the social dimension of learning--think vicarious observation \citep{Madsen2018}. The literature distinguishes between two dimensions--reliablity and validity \citep{March1991a}. Reliability describes how widely shared a model of the world has become. Validity describes how useful the model is for prediction and control \citep{Rerup2021}. Contrary to our intuition, models that are suboptimal in terms of validity can become widely shared in the world \citep{Levitt1988}. The third chapter is dedicated to the theory of reliability and validity.