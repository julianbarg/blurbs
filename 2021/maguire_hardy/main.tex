\documentclass{article}

\usepackage[natbib=true, style=apa]{biblatex}
% Put float first to avoid some warning
\usepackage{float}
% To use toprule
\usepackage{booktabs}
\usepackage{caption}
\usepackage{tabularx}

\addbibresource{bibliography.bib}

\title{Maguire \& Hardy (2009) timelines}
\author{
	Barg, Julian\\
	\texttt{jbarg.phd@ivey.ca}
}

\sloppy

\newcommand{\timeline}{\hspace{-2.3pt}$\bullet$ \hspace{5pt}}
\newcommand{\tabincell}[2]{\begin{tabular}
{@{}#1@{}}#2\end{tabular}}

% Make is so only year is printed
\AtEveryBibitem{
  \clearfield{labelmonth}
  \clearfield{labelday}
  \clearfield{labelendmonth}
  \clearfield{labelendday}
}


\begin{document}
	\maketitle

	\newpage

	\begin{table}[H]
		\caption{DDT events}

		\begin{tabularx}{\textwidth}{r @{\hspace{0.5\tabcolsep}} l |@{\timeline} X}
			\toprule

			1874 & & DDT discovered.\\

			1939 & & DDT insect-killing properties are discovered.\\

			1941-1945 & & DDT used to protect US troops from insect-borne diseases in WW2.\\

			1945 & & The production of DDT for the military, now obsolete, is employed to benefit domestic agriculture.\\

			1947 & & Federal Insecticide, Fungicide, and Rodenticide Act is passed. Requires manufacturers to test and report product safety to government.\\

			1952 & & The question of the safety of DDT is touched on in \textit{Nature}, authors state that no ill-effects have been reported yet \citep{Davidson1952}.\\
	 
			1959 & & Production of DDT peaks. DDT is the top-selling insecticide. At the same time, the literature has already picked up risks and exposure to DDT.\\

			1962 & & Rachel Carson publishes \textit{Silent Spring}.\\

			1963 & & \textit{Nature} and \textit{Ecology} review \textit{Silent Spring}.\\

			1963 & & The President' Science Advisory Committee (PSAC) calls for an end of the use of DDT and other pesticides.\\

			1968 & & Wisconsin classifies DDT as a water pollutant.\\

			1939 & & Various states implement bans of different uses of DDT.\\

			1971 & & A court orders the EPA to hold hearings on DDT.\\

			1972 & & The EPA bans DDT nationwide.\\

			1972 & & DDT usage has already declined by 67\% relative to 1962 levels.\\

			1973 & & The ban comes into effect.\\

		\end{tabularx}

	\end{table}

	\begin{table}[H]
		\caption{DDT discourse}

		\begin{tabularx}{\textwidth}{r @{\hspace{0.5\tabcolsep}} l |@{\timeline} X}
			\toprule

			1959 & Ecology Textbook & Effectively no mention of DDT.\\

			1962 & Entomology Textbook & Acknowledge potential adverse impacts of DDT in food.\\

			1963 & Government Report & Mention of massive losses of birds and fishes, and potential dangers to humans.\\

			1966 & Government Report & Vaguely mentions risks to birds.\\

			1969 & Government Report & Again vaguely mentions risks to birds and other species.\\

			1971 & Entomology Textbook & Acknowledges storage in fat and accumulation process--biological magnification. Textbook mostly more detailed.\\

			1971 & Ecology Textbook & DDT poisoning of food chains introduced.\\

		\end{tabularx}

	\end{table}

	\begin{table}[H]
		\caption{Regulative action}

		\begin{tabularx}{\textwidth}{r @{\hspace{0.5\tabcolsep}} l |@{\timeline} X}
			\toprule

			1964 & & Pesticide law reform allows for easier cancellation of registration.\\

			1970 & & Creation of EPA--often directly attributed to \textit{Silent Spring}.\\

			1972 & & Ban on DDT, on grounds of substitutability--more benign pesticides.\\

		\end{tabularx}

	\end{table}

	\begin{table}[H]
		\caption{Industry pushback}

		\begin{tabularx}{\textwidth}{r @{\hspace{0.5\tabcolsep}} l |@{\timeline} X}
			\toprule

			1962 & & Industry associations and companies push back in pamphlets and reviews against \textit{Silent Spring}.\\

			1962 & & \textit{Time} defending DDT on ground of National Academy of Science record.\\

			1963 & & Company representative pushing back on national TV.\\

			& & (Various more).\\

			1969 & & WHO director-general defends DDT usage for purpose of fighting epidemics--utilitarian logic.\\

		\end{tabularx}

	\end{table} 

	\clearpage
	\printbibliography

\end{document}