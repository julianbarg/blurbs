%!TEX root = ../main.tex

% \begin{itemize}
% 	\item Hone in on literature that theorizes discourse, not just mentions it
% 	\item \#ToDo: Add definition of discourse
% 	\item Pivot to Maguire \& Hardy
% \end{itemize}

\textit{Discourse} is not an imposing concept. The name of the accompanying method--\textit{discourse analysis}--pretty much speaks for itself. A layman's understanding of "discourse" is not too far off (except for the way that Foucault uses the term). The word \textit{discourse} is often casually used in the OT literature. And qualitative empirical work, such as \textit{frame analysis}, often constitutes a sort of discourse analysis. But few are talking \textit{specifically} about discourse as a concept or process. In the OT literature Maguire and Hardy hold a virtual monopoly on \textit{recent} work: there are two dormant conversations--one in \textit{Organizations} \citep[e.g.,][]{Burrell2000} and one in \textit{Organization Science} \citep[e.g.,][]{Grant2004}.

% \begin{itemize}
% 	\item Maguire \& Hardy--basics
% 		\subitem Work on chemicals
% 		\subitem Specific derivation of discourse analysis, NOT universal/open
% 		\subitem Accomplishments in introducing discourse/social construction
% \end{itemize}

Most works of Maguire and Hardy follow a similar pattern, with only slight changes over time. The articles shine light on the discourse around chemicals--specifically, DDT, BPA, and VAM. The articles usually take after the work of \citep{Fairclough1992}, who attempted a synthesis of authors as diverse as Foucault, Giddens and Habermas. In the front end, their papers generally provide some form of introduction to discourse, beginning with very basic concepts in the early years. For instance, the first empirical paper introduces front and center the \textit{social construction} of \textit{objects} through discourse, and the crucial role of this process for the formation of institutional fields \citep{Hardy1999}.

% \begin{itemize}
% 	\item Primacy of text in M\&H
% 	\item Emphasis of \citet{Phillips2004} as main instance of theorizing
% 	\item Foreshadowing: power and politics concern
% \end{itemize}

A full discussion of the theoretical underpinnings of this current strain of discourse analysis in the mainstream OT literature can be found in \citet{Phillips2004}. The article highlights the importance of language for the formation of institutions. Institutions, so the claim, cannot be understood without an analysis of the accompanying language in the form of "a discourse that constitutes [the institution]" (p. 647). The article holds a promise: find me the appropriate body of corpus of texts, and I can show you the world. And in the conclusion a preview of things to come: an analysis of texts and their fate--either being ignored or becoming canon--"reconnects institutional theory to a concern with power and politics" (p. 648).

% \begin{itemize}
% 	\item Examples of empirical work
% 	\item Demonstrate studying of texts \& consequences
% 	\item Demonstrate primacy of "officially sanctioned" text
% 	\item Exemplarize level of analysis: concepts/actions
% \end{itemize} 

Two examples of their early work shall illuminate the kinds of texts usually selected for analysis \citep{Maguire2006,Hardy2010}. The articles analyze the Stockholm Convention on Persistent Organic Pollutants. As with their other works, Maguire \& Hardy focus on the specific discursive event, with most data being either conference documents, or documents that directly influence the debate. The coded data reveals a discursive construction process that is bound in time and space. The contribution stems both from empirically demonstrating the process of social construction of the new object through discourse, and from shedding additional light on specific concepts within the discourse literature. For instance, \citet{Maguire2006} demonstrates how the present and the past are intertwined in discourse. Actors either \textit{reconcile} their standpoint or with legacy discourse, or they \textit{challenge} the existing discourse--regardless, past discourse acts as a reference point and new texts do not stand in a vacuum. Similarly, \citet{Hardy2010} focuses on the role of \textit{narratives}.

% \begin{itemize}
% 	\item Selection of material
% 	\item Criticism by other qualitative researchers
% 	\item Translation in 2009 \& 2020 and evolution of analysis 
% \end{itemize} 

At the same time, \citet{Lok2006} point out that the analysis of language is not followed all the way through--we shall return to their comments later. In Maguire \& Hardy's empirical work, there is an--arguably necessary--selection taking place. What makes their work stand out\footnote{Especially in their journal of choice, AMJ.} is their resolute focus on texts, and the entirely empirical demonstration of concepts, that allows for their work to stand side by side with quantitative work \citep[cf.][]{Phillips2004}. Over time, they have included a broader array of materials in their work, as a comparison between two of their articles on \textit{translation} shall demonstrate \citep{Maguire2009,Hardy2020}.