%!TEX root = ../main.tex

\subsection*{Production of Texts}

Over time, Maguire \& Hardy redirected the attention of their empirical inquiry toward interaction between many actors. Yet, the conceptualization of institutional work processes underlying Maguire \& Hardy's empirical work--\citet{Phillips2004}--is incomplete \citep{Lok2006}. Their approach to discourse analysis applies a constructivist lens only to the production of knowledge on objects. When attending to texts, Maguire \& Hardy tend to switch to a "realist mode":

\begin{quote}
	"Symptomatic of this realist mode is the absence of any indication of (reflexive) awareness of how the identification of actions, texts, and so forth is itself an articulation of specific (and contested) language games. Instead, key components of Phillips et al.'s 'discursive model of institutionalization' [...] are represented as if they are immediately accessible" \citep[p. 478]{Lok2006}.
\end{quote}

In the empirical work on discourse, \citet{Maguire2009} provides a lonely glance of what the constructivist mode would look like in the empirical analysis of texts. After Carlson publishes \textit{Silent Spring}, the chemical industry publishes various countertexts that try to disrupt Carson's participation in the discourse. These countertexts are not limited to \textit{challenging} the validity of Carson's statements in her book \citep[cf.][]{Maguire2006}. Rather, they challenge the legitimacy of Carson's participation in the discourse. Carson's "allegations" meet industry "facts" and her words should not be taken "at face value" \citep[p. 165]{Maguire2009}. Concerns are raised about the "emotional outbursts" in her book and because she is a women \citep[p. 166]{Nicolini2009}. These findings speak to the taken-for-grantedness of the institution, ad how it is defended \citep{Steele2021}. Arguments are not limited to \textit{challenging} and \textit{reconciling} \citep{Maguire2006} but concern the admissibility of accounts into the discourse.

