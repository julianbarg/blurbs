%!TEX root = ../main.tex

\subsection*{Translation and Chains of Actors}

% \begin{itemize}
% 	\item Translation
% 		\subitem \citet{Latour1984,Latour2005b}
% 		\subitem Relate to zooming out to "star-shape" \citep{Nicolini2009}
% \end{itemize}

\textit{Translation} is a great example both of the merit of Maguire \& Hardy's work, and on opportunities for extension, as translation is a central concept in Actor-Network Theory. It describes how a sign or token--such as "DDT"--does not take its meaning from an initial power or force with enough inertia for it to "survive" rounds of transmission. Rather, the meaning of the sign is derived from the entirety of translations that have taken place. Each and every participant in the chain of translations is an actor who has a part in the perpetuation and inevitable alteration of the sign \citep{Latour2005b}. For instance power is not constituted by the existence of the "power\textit{ful}", an individual who "hoards" power. Rather it is the many individuals who act on command--and while doing so interpret the order--who allow for power to exist. Power, like other signs, is constituted by its application and cannot be directly stored \citep{Latour1984}.\footnote{This derivation of power is the purpose of Latour's original work on the concept.} The result is a \textit{star-shaped} \citep[cf.][]{Nicolini2009} practice as connected individuals share common understandings of signs that can be identified through research.

% \begin{itemize}
% 	\item \citet{Maguire2009} vs. \citet{Hardy2020}
% 	\item \citet{Maguire2009}
% 		\subitem Linear process, everybody reacts to perceived sole agent Carson
% 		\subitem Character of BBP is a derivation of Carson's account, altered
% \end{itemize}

\citet{Maguire2009} emphasizes the transformative character of translation. The article begins to introduces the notion of a negotiated understanding of DDT. The final understanding is shown not to be a direct adoption of the findings in Rachel Carson's \textit{Silent Spring}. \textit{Silent Spring} problematizes multiple issues concerning DDT and pesticides more generally. As experts in other areas investigate those problematizations, some persist, relatively unscathed, while others are dropped or altered. Rachel Carson's claim regarding the dangers for human health for instance is qualified--its negative impacts are subsequently regarded as neither proven, nor disproven. \citet{Maguire2009} only begins to shake the image of the "hypermaskular" \citep[cf.][]{Suddaby2017} institutional actor. The agency of others only comes into play when they interpret her accounts. 

% \begin{itemize}
% 	\item \citet{Hardy2020}
% 		\subitem \textit{Multinodal}, parallelized process
% 		\subitem Multiple things are happening at multiple places, sometimes simultaneously
% 		\subitem All these things constitute character of BPA, no clear actor/none highlighted
% 		\subitem Getting closer to the conception of a "chain [that] is made of \textit{actors}" \citep[p. 269, emphasis in original]{Latour1984}
% \end{itemize}

In contrast--and while the conceptualization of translation in the front end changes relatively little--\citet{Hardy2020} demonstrates the simultaneous investigation into BPA at multiple points in a network, by researchers from different disciplines, governments, and businesses in different countries. The translation process here is an inherently social one, where multiple events occur at multiple places at the same time. Toxicologists question the notion that the "dose makes the poison", while endocrinologists research the potential for BPA to interact with cellular hormone receptors, and Canadian retailers reconsider their reputational risk and the potential backlash from consumers, returning products in the thousands. In the words of \citet{Latour1984}: "the chain is made of \textit{actors}" (p. 269, emphasis in original).

% Taking serious alternative explanations--why do they not hold true?

% \begin{itemize}
% 	\item Use \citet{Lok2006} to pivot to politics
% 	\item Go beyond Maguire \& Hardy's recent work by including the creation of texts/discourse
% \end{itemize}