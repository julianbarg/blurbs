%!TEX root = ../main.tex

\subsection*{Empirical context}

Before the Deepwater Horizon oil spill in 2010, the construction of pipelines in the US went over virtually unchallenged. Keystone and Keystone XL are a useful yardstick: the approval process for the original \textit{Keystone} pipeline from 2005-2009 garnered almost no public attention, whereas the approval process for \textit{Keystone XL} was drawn out from 2008-2019, and in 2021 the permit was canceled. While in 2009, there is one hit for the Keystone pipeline\footnote{Combined keywords "Keystone", "pipeline", and "TransCanada" via Factiva.} in the ten major national newspapers in the US, in 2011 there are 105.\footnote{Google Trends shows a similar trend, see \url{https://trends.google.com/trends/explore?date=all&geo=CA&q=\%22Keystone\%20pipeline\%22}.} Keystone XL in particular became a highly symbolic project, both for the oil and gas industry, and for its opponents. The oil and gas industry is concerned that the defeat of Keystone XL will embolden opponents of fossil fuels \citep{Freitas2021}. For the opponents of Keystone XL, the pipeline has become a symbol for the recklessness of the fossil fuel industry--Keystone XL was intended to carry oil sands, which also started to come under fire for their environmental impacts at around 2010 \citep{Schindler2010}.

Since the onset of this challenge to the legitimacy of pipelines, two notable pipeline projects have been attempted. First, there is Keystone XL itself. Keystone XL has been challenged on the national, state, and local level. It was a repeated target of demonstrations in Washington, and subject of debate in the 2015 and 2020 Presidential elections. Environmental NGOs raised challenges against the pipeline throughout the approval process in all three states that were to be crossed. And at the local level, activists engaged in blockades against material and equipment.

The Dakota Access Pipeline (DAPL) faced fewer regulatory challenges. The approval process was swift, since DAPL does not cross any international border. In contrast, the physical resistance against DAPL was much more fierce, culminating in a showdown in South Dakota that has become known as the "Battle at Standing Rock" \citep{Read2016}. As of 2021, DAPL is transporting oil across the Great Prairies, but has been caught up in legal battles with environmental interest groups that have more than once come close to shutting down the pipeline.

In addition to these big projects, there were a number of smaller projects that attracted resistance. These include Enbridge Line 5, which is under threat of being shut down by the government of Michigan. The shutdown would indicate a new quality of enforcement action against existing pipelines that are believed to be unsafe. Enbridge Line 3 has been caught up in the storm of resistance that brewed up against Keystone XL--Enbridge originally intended to increase the capacity of the line by adding a new segment, but so far the efforts have been halted by Minnesotans. Finally, the American Petroleum Institute (API) is broadly involved with pipeline projects. For the API, the woes of pipeline operators are harbinger of challenges that linger at the horizon for the oil and gas industry as a whole. Sometimes, the API gets involved with specific pipeline projects, often the API lobbies for pipeline, and oil and gas projects in general.