%!TEX root = ../main.tex

\subsection*{Actors}

The purpose of this write-up is to present a selection of the strategic responses of the pipeline industry and their allies to the legitimacy threat that appeared in 2010. The criteria for inclusion is that the events covered in one of the ten largest newspapers in the US. The responses range from direct--preventing pipeline opponents from voicing their opinion--to indirect attempts to influence opinion leaders.

The events highlight that what we think of as the "institution" is a conglomerate of diverse actors. Similar to \citet{Montgomery2020b}, the motivations of new and old "custodians" vary widely, but in aggregate, they pull in roughly the same direction. Keystone XL and the Dakota Access Pipeline have taken on a broad and big meaning--far beyond the financial interests associated with their construction and operation. The pipeline projects have become synonymous with a model of modernity that is built on the exploitation of natural resources for economic prosperity. The conflict over the pipelines is an almost perfect metaphor. A pragmatic industry, driving America forward against the wishes of liberal demagogues. \textit{Note: this should be grounded in the qualitative data.}

The coalitions of pipeline "custodians" is as broad as their ideological base. There are of course the pipeline operators themselves, but many others also act on behalf of these operators. 
There is the small private security firm TigerSwan that is a bit too proactive in generating strategies against pipeline protesters, and drawing on tactics that employees have picked up in Iraq and Afghanistan. 
There is the state of North Dakota, which willingly spends \$43 million for police forces to make protesting against the Dakota Access Pipeline as unpleasant as possible. The state legislature also adjusted its state law to increase the legal repercussions for protesters. And finally, the American Petroleum Institute, which sponsored a trip to Alberta to facilitate meetings between American lawmakers and representatives of TransCanada.
