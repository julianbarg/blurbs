%!TEX root = ../main.tex

Recently, there have been complaints by institutional theorists about strains of the literature that showcase \textit{heroic} actors who all too easily change institutions, \textit{singlehandedly} \citep[e.g.,][]{Suddaby2017}. These strains of literature, they argue, import ideas from other fields and thereby destroy the  theoretical assumptions of institutional theory \citep{Suddaby2010}. In other words, institutions should be mighty and unwieldy, towering over the actor.

The \textit{procedural} or \textit{discoursive} strain of institutional theory draws on the work of e.g., \citet{Maguire2009}, who highlight the agentic actions of Rachel Carson taking on DDT and, by extension, the American chemical industry \citep{Suddaby2017}. Certainly, giving the impetus for the deinstitutionalization of DDT is not a small feat and should be appreciated. But at the end of the day, Suddaby argues, institutional change is typically "an inherently distributed effort of diverse change agents at multiple levels who engage in the day-to-day effort of legitimacy work" \citep[p. 462]{Suddaby2017}.

\citet{Maguire2009} do let it shine through though that it \textit{has not} been that easy for Rachel Carson. Before she succeeded in her noble goal of protecting the environment and people from a dangerous pesticide, she had to suffer through a barrage of attacks from the industry on her hypotheses and her person. So how easy or difficult is it changing and institution? \citet{Oliver1991} suggests a plethora of weapons an organizations can wield in response to institutional threats. 

% Now we are getting somewhere. 
There are two possible responses to Suddaby when he laments the "hypermaskular" \citep{Suddaby2010,Suddaby2017} actor. One possibility would be to investigate what happened \textit{after} Rachel Carson published her book. To identify her multipliers. And to showcase the texts that constitute the day-to-day work of delegitimization.\footnote{Aka the second chapter of my dissertation.} Another route is to take a second look at her counterparts. Who are the institutional defenders? What steps do they take to protect the legitimacy of the institution? That is the second route. We counter the image of institutions towering over actors, with an image of two larger-than-life actors--or sometimes an actually hopelessly out of his league David and a Goliath--locked into an epic fight over the legitimacy of an institution.

Notably, \citet{Oliver1991} laid the groundwork for this inquiry into institutional actions. Oliver noted that organizations sometimes acquiesce, and compromise, but they may also engage in avoidance, defiance, and manipulation when they sense a threat. More recently, the "agency heavy" literature includes \citet{Suddaby2005} analyzing how one pioneering bank changed the typical structure of the Big Five accounting firms by purchasing a law firm. Or \citet{Delmestri2016} showcasing how an Italian distiller successfully recategorized its product to fit a high status niche. And as a final example, \citet{Navis2010} study the establishment of satellite radio as a new market category. True to the nature of the \textit{procedural} or \textit{discoursive} strain of institutional theory, these are all examples of institutions in flux--not to our surprise we find examples of examples that kick off significant change.

Where would we look for examples of "the other side" of institutional change? A preferred choice would be a context where the threat to legitimacy is salient, and the institutional incumbents have sufficient organizational resources--experience, network, and finances aka \textit{power}--to face off against the threat. In the next section, I argue that the pipeline industry is one such context, and that there are multiple suitable events to study in that context. 

Using this approach of theorizing based on a very salient case, we have to live with the obvious disadvantage that it provides us with a view of the world that is not just "hypermaskular", but "hyperaggressiv". Just like the rather plain descriptions by \citet{Heracleous2001} of the London Insurance Market's early venture into electronic risk placing are probably not representative of the great lot of important processes that we try to capture with institutional theory \citep{Hoffman2015}, so do these ones from the pipeline industry provide a skewed view. I would concur that while the case of the pipeline industry may be exceptional in its intensity, it is representative of processes that matter--the viciousness of the discourse on an institution may be a good indicator that it is a process worth paying attention to \citep[cf.][]{Ergene2020}.
