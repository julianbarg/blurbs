\documentclass{article}

\usepackage{hyperref}

% To left-align footnotes if want be.
% \usepackage[hang]{footmisc}

\title{Thematic Opening}

\author{
	Barg, Julian\\
	\texttt{jbarg.phd@ivey.ca}
}

\sloppy

\begin{document}
	
	\maketitle

	\subsection*{Keystone pipeline system}

	They were supposed to a fabulous pair of twin pipelines. The first, Keystone, began conduction in 2008 and was finished in three sections between 2010-2014. They were supposed to solve an enviable problem of TransCanada. TransCanada had access to a great amount of resources in Alberta--namely, Alberta's oil sands--but it had trouble getting these resources to the market. Sometime in the 2000s, TransCanada realized that going South through the US would be easier than going West, through the Rockies in British Columbia. So TransCanada repurposed existing pipelines in Saskatchewan and Manitoba, and then built a new line through North Dakota, South Dakota, Nebraska, Kansas and Oklahoma. In the South, TransCanada could make use of the already existing refineries and infrastructure of Texas to refine the diluted product and sell it at the Gulf of Mexico.

	TransCanada encountered little resistance when applying for a permit from the Bush administration. And so in June 2008, three month after obtaining the Presidential Permit, TransCanada had an even better idea. It could increase its export capacity even further by building a second pipeline--Keystone XL--straight throuh the US, without going East along the Canadian border first. The new pipeline would go straight across Montana and South Dakota, and meet the first pipeline in Steele City, Nebraska.

	The permit application process, and--for the most part--the construction of the first pipeline, Keystone, encountered relatively little resistance. TransCanada could not have anticipated that an event in 2010 would change that. In 2010, an explosion occurred on the Deepwater Horizon offshore drilling rig in the Gulf of Mexico. The platform subsequently spilled 4.9 million barrels of crude into the Gulf. The magnitude of the spill can only really be appreciated when viewed from space--even relative to the Mississippi River Delta, it is large.\footnote{\url{https://en.wikipedia.org/wiki/File:Deepwater_Horizon_oil_spill_-_May_24,_2010_-_with_locator.jpg}.} These pictures of the spill--and of course those of animals being cleaned of oil--sent the public opinion on the oil industry on a new trajectory.

	TransCanada began to feel more headwind after that. While Obama personally voiced support for Phase III of the original Keystone pipeline in 2012, his Department of State rejected two separate Keystone XL permit application--one in 2012, and another one in 2015. Trump invited TransCanada to resubmit its application, and signed the Presidential Permit in March 2017, just three months after taking office. However, the project now faced legal headwind because of the rushed nature of the new permit. Trump tried to rectify that problem by canceling the problem-ridden permit and issuing a new one in 2019. When Biden took office on in 2021, construction had not yet begun. Biden canceled the permit on the day he took office.

	\subsection*{The bigger picture}

	Clearly, concerns about the safety of the pipeline and about the oil and gas industry by and large had persevered? TransCanada's claims that the pipeline would be ten times as safe as conventional pipelines were unbelievable at best. However, the track record of Keystone XL in the regulatory arena suggests otherwise. TransCanada obtained all necessary permits in all three states, and never once did the opposition pose any serious threat below the federal level. Year for year, critics showed up for hearings on the permit applications and reapplications. But even when critics turned to state supreme courts, their successes were marginal at best. And Obama--famously a former law professor at University of Chicago--rejected the application in 2012 because he was concerned that the assessment process would not hold up, not because of environmental concerns. In 2015, his State Department cited concerns about the credibility of the administration's international climate change efforts as the reason for a negative National Interest Determination. Trump's efforts to rubber stamp the project only failed because of his ineptitude as an administrator. The Dakota Access Pipeline (DAPL)--a similarly colossal and controversial project--in 2015 faced no issues and was completed in 2017. DAPL passes through North Dakota, South Dakota, Iowa and Illinois, and did not require a Presidential Permit because it does not cross a national border. Finally, the Biden administration decided that it will not shut down the Dakota Access Pipeline over the lack of an Environmental Impact Statement.\footnote{\url{https://www.ogj.com/general-interest/article/14202049/biden-administration-allows-dakota-access-to-keep-operating-for-now}.}

	Remains the question, what has changed after Deepwater Horizon? The oil and gas industry certainly has not suffered--between 2010-2019, oil production in the US has more than doubled, from about 7,500 barrels to over 17,000 barrels per day.\footnote{\url{https://www.bp.com/content/dam/bp/business-sites/en/global/corporate/pdfs/energy-economics/statistical-review/bp-stats-review-2020-full-report.pdf}, p. 16. Via \url{https://www.statista.com/statistics/265181/us-oil-production-in-barrels-per-day-since-1998/}.} The pipeline industry is also alive and well, having increased its footprint from about 180,000 miles to about 225,000 between 2019-2019.\footnote{\url{https://www.bts.dot.gov/content/system-mileage-within-united-states} via \url{https://www.statista.com/statistics/197932/us-pipeline-system-mileage-since-2004/}.}

	Then what is left of the "triumphant" narrative I offered in the first part of this write-up? We certainly see a rise of activism over time. Catalyzing events such as the Deepwater Horizon oil spill, the anti-Keystone XL demonstrations outside the White House, or the Dakota Access Pipeline protests near Standing Rock have made more people aware of the dangers and safety concerns surrounding the oil and gas industry. The activism against pipelines, be it on the street or in state assemblies, has also allowed anti pipeline proponents to build and grow their networks. For instance, we have seen the unlikely alliance between the TransCanada whistleblower and environmental activists in South Dakota. And we have seen that a democratic president in the White House is a potential--not a definitive--ally to pipeline activists.

	What I have described in the above paragraph is the emergence of new epistemic communities, it is \textit{not} otherwise a shift in mentality. On would be easily misled into believing that the events of the last decade represent a fundamental shift in social structure. However, the oil and gas industry is still alive and well, and so are their networks. The Koch family foundations still wield a lot of influence, and so does the American Petroleum Institute. Support for pipelines has not diminished in the three states that were to house Keystone XL. And of course there is the brutal crackdown on the protest against the Dakota Access Pipeline near Standing Rock, for which the state government of North Dakota mobilized the National Guard. Finally, if we were to talk sheer numbers, the number of protesters would have to be compared to the number of engineers, workers, and policemen who support the oil and gas industry directly or indirectly.

	Compare the narrative offered here to Maguire and Hardy (2009), and a more complex picture of discourse and social change emerges. Diverging voices win on one issue--they win the battle, but not the war. Institutional change is accomplished by a new social alliance, but this new structure does not supersede the existing structure. We observe the emergence of a new discourse--or new discourses if we are to divide out the manifold issues that are raised-- which now coexists alongside the preexisting discourse. These discourses are represented by their corresponding epistemic communities. The overall process corresponds to a social stratification or polarization, not to a resolution of environmental concerns.

\end{document}