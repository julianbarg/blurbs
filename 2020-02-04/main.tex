\documentclass{article}

\usepackage[natbibapa]{apacite}
% \usepackage{paracol}
\usepackage{todonotes}

\title{Realistic Environmentalism}
\author{Julian Barg}

\begin{document}
	\maketitle

	\section*{Introduction}

	\begin{quote}
		\itshape
		A personal episode

		Before I wrote my first term paper I had learned to avoid words such as "should", "must" or "have to". I had learned that the researcher should avoid normative language and stay "neutral". Today we know that an author's background inevitably enters the writing process, if not in language then in content, and that sometimes it is more sensible for us reflect on our values and maybe even "disclose" them to our reader. I have come to realize that the abovementioned rule of thumb still holds true in one sense though: it is not that we have to keep our value or norms or opinions out of our research because our research needs to be "objective". Rather, the hard truth is that if you tell "the world" in your paper what to do, the world probably will not listen to you. "The world" probably will not care--beyond your readers.\footnote{And in some cases even those will not care.} Sometimes, it is more sensible for us to lay out the facts and let the audience arrive at the conclusion herself or himself.

		See also \citet{Gouldner1962,Zbaracki2021-02-03,Jones2019,Boisot2010}.
	\end{quote}

	In terms of citations and cultural impact, the last 30 years have been an unlimited success for sustainability research. Yes, the community is less than satisfied with the outcomes \cite[e.g.,][]{Ergene2020}. This article takes a look at the original motivation of the research and how that motivation has influenced the research that happens under its umbrella \citep{Latour1987}. That insight on the motivation and the research process allows us to problematize missed opportunities \citep{Alvesson2011}. Toward this goal we carry out a discourse analysis.

	% This article creates a realistic theory\todo{May also call it a research agenda.} for research on organizations and the natural environment. The suggested approach is based on insights of what has and has not worked for sustainability research to date. This includes taking a critical look at sustainability research and how it has affected the real world, since that is the benchmark for normative research. Hence, this article is borrows from developments in sociology--namely science and technology studies. The concepts of \textit{valid} and \textit{reliable} knowledge illuminate the underlying issue. In order for knowledge to leave an impact it must not only be useful for prediction and control of future events (valid), but it also needs to be shared among a sufficient share of people in an organization or society (reliable).

	\begin{itemize}
		\item Insert paragraph outlining findings on sustainability research \& practice stagnation.
	\end{itemize}

	% A \textit{realistic environmentalism} suggest possible approaches to tackle/overcome/create

	% could allow us to overcome 30 years of standstill.

	% More powerful powerful research 

	% Our suggestion is for research on organizations and the natural environment to adopt a more \textit{realistic} perspective on the environment. This \textit{realistic approach} involved (1) research 
	% (2) Research that problematizes . (3) A relational approach 

	% To learn from past research article arrives at a research agenda advocates for more "realistic" 

	% Based on these three points, our research agenda postulates that (1) research should focus on specific local 
	% rather than an imagined 

	\clearpage
	\bibliography{bibliography}
	\bibliographystyle{apacite}

\end{document}