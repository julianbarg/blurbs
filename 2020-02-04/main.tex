\documentclass{article}

\usepackage[natbibapa]{apacite}
% \usepackage{paracol}
\usepackage{todonotes}
\usepackage{hyperref}

\newcommand\wikilink{\href{http://wiki.jbarg.net/Constructivism\%20\%26\%20Ecology.md\#environmental-science-stream}{link}}

\title{Realistic Environmentalism}
\author{Julian Barg}

\begin{document}
	\maketitle

	\section*{Introduction}

	\begin{quote}
		\itshape
		A personal episode

		Before I wrote my first term paper I had learned to avoid words such as "should", "must" or "have to". I had learned to avoid normative language and stay "neutral". Today we know that an author's background inevitably enters the writing process, if not in language then in content, and that sometimes it is more sensible for us reflect on our values and maybe even "disclose" them to our reader. I have come to realize that the abovementioned rule of thumb still holds true in one sense though: it is not that we have to keep our value or norms or opinions out of our research because our research needs to be "objective". Rather, the hard truth is that if you tell "the world" in your paper what to do, the world probably will not listen to you. "The world" probably will not care--beyond your readers.\footnote{And in some cases even those will not care.} Sometimes, it is more sensible for us to lay out the facts and let the audience arrive at the conclusion herself or himself.

		See also \citet{Gouldner1962,Zbaracki2021-02-03,Jones2019,Boisot2010}.
	\end{quote}

	In terms of citations and cultural impact, the last 30 years have been incredibly successful for research on organizations and the natural environment. Yet, the community is less than satisfied with the outcomes \cite[e.g.,][]{Ergene2020}. This article takes a look at the motivations that drive sustainability research and the imprint these motivations have had on the research \citep{Latour1987}. That insight on the motivation and the research process allows us to problematize missed opportunities \citep{Alvesson2011}. Toward this goal we carry out a discourse analysis.

	At the heart of sustainability research, there are a number of debates that signify distinct epistemic communities, each forming around a disparate larger consensus \citep[cf.][]{KnorrCetina2000}. First, there is a rift between ecocentrism and the environmental management paradigm \citep{Purser1995}. Second, there is a divide between research focusing on solutions, and research describing problems. The divide between problems and solutions is related to strategizing among researchers with regard to hypothesized effects on the reader \citep{Westoby2019}. And finally, there is an underlying disagreement on strategies. Some papers suggest small, gradual changes, while other papers argue that only sweeping changes are suitable for addressing problems such as an accelerated global climate change. Underlying these discourses is also a direct effect the research has on the researchers that is described as environmental or climate grief \citep{Conroy2019,Cunsolo2018}.

	Aiding the discussion will be the two constructs of reliability and validity. Organizational learning is said to be valid when learning outcomes can be effectively used in prediction and control. 
	% For example, the research of the IPCC is valid, as it produces correct prediction on temperature and weather changes. 
	Organizational learning is reliable when the outcome of the learning process is shared across the members of the organization or a collective \citep{Rerup2021}. 
	% Climate denial in the Trump administration is an example of lack of reliability of knowledge. 
	Validity and reliability represent a challenge for research on organizations and the natural environment: what may have been found to be unsustainable in the academic literature can still be a viable practice
	\todo{This is where the empirical works I collected from environmental science on the social construction of the environment comes into play (\wikilink).}
	e.g., in the economic system \citep{Habermas1984}.
	\todo{Did not check whether this is the right volume.}

	The discourse analysis on the three debates--especially that on ecocentrism and environmental management--reveal that there is little hope for a theoretical discussion to arrive at a definite answer. They represent diverging hypotheses about a future that has not arrived yet. Comparison to other fields and debates suggest possible pathways for resolving the deadlock \citep[cf.][]{Giddens1979a}. Specifically, the comparison suggest that the debate could move on by (1) discussing specific ecosystems rather than "nature" as a universal construct. (2) Research that problematizes rather than primarily offering solutions. (3) By taking a relational approach to integrate different streams.

	% This article creates a realistic theory\todo{May also call it a research agenda.} for research on organizations and the natural environment. The suggested approach is based on insights of what has and has not worked for sustainability research to date. This includes taking a critical look at sustainability research and how it has affected the real world, since that is the benchmark for normative research. Hence, this article is borrows from developments in sociology--namely science and technology studies. The concepts of \textit{valid} and \textit{reliable} knowledge illuminate the underlying issue. In order for knowledge to leave an impact it must not only be useful for prediction and control of future events (valid), but it also needs to be shared among a sufficient share of people in an organization or society (reliable).

	% Our suggestion is for research on organizations and the natural environment to adopt a more \textit{realistic} perspective on the environment. This \textit{realistic approach} involved (1) research 
	% (2) Research that problematizes . (3) A relational approach 

	% Based on these three points, our research agenda postulates that (1) research should focus on specific local 
	% rather than an imagined 

	\clearpage
	\bibliography{bibliography}
	\bibliographystyle{apacite}

\end{document}